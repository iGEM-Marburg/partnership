\epigraph{The following chapter was written by Lennart Lutz \& 	Paul Goffing}{\textit{iGEM Bielefeld-CeBiTec 2021}}

\section{Motivation \& Scope} 
\textbf{Welcome dear iGEMer!}
\newline\newline
In this chapter we would like to explain to you what the iGEM Phototrophs handbook is all about. This handbook was created during a collaboration of several iGEM teams from all over the world, which worked together to create a global community of teams working with phototrophs. Since working with phototrophs inhabits some challenges, which differ from the work with heterotrophic organisms like bacteria or yeast, we worked together to identify these challenges and to try to overcome them. To organise our exchange of ideas and support, the iGEM Teams Marburg and Bielefeld-CeBiTec created a community called “iGEM Phototrophs - Overgrow the World”. This community consists of three main parts: a slack workspace to connect and communicate, a series of four meet-ups with interesting guest speakers from phototrophic synbio and the work on this handbook. The first two parts, the slack workspace and the meet-ups, were mainly meant to help out each other with their current project. The handbook is meant to transport this year’s experiences to the future generations of phototrophic iGEM teams.
\newline\newline
What we envision for the future of this iGEM Phototrophs Handbook is to be a living document, to be expanded and revised by future iGEM teams and to be a resource for help and advice for iGEM teams working with phototrophic chassis. Here we explicitly include all three groups of phototrophic chassis: cyanobacteria, algae and plants. Now, one might wonder, how can this handbook help you? Well, at first, we included information about what you should consider, if you want to work with a phototrophic chassis. Second, we included many experiences this year’s teams made with their chassis, regarding topics like acquiring organisms, cultivation, transformation and cloning. Third, we collected an almost complete list of teams which worked with phototrophs since 2015, so that you can get inspiration, ideas or find out how they solved their problems. To cover as many aspects as possible, 
\newline\newline
Many of the iGEM teams working with phototrophs from all over the world are part of our community and contributed to this handbook:

\subsubsection*{ASU - United States}
\paragraph{Genetically engineering the microalgae Chlamydomonas reinhardtii to sequester arsenic from contaminated groundwater}\mbox{} \\
Arsenic contamination in groundwater is a serious problem both in local Arizonan communities and abroad: prolonged exposure to arsenic contamination can cause cancer, vascular damage, and liver failure. This project aims to engineer the microalgae Chlamydomonas reinhardtii to sequester arsenic out of water. Metallothionein, arsenate reductase, and ferritin were integrated into the microalgae via the pASapI plasmid in varying permutations. The plasmid rescues function of the photosystem II gene, leveraging the ability to photosynthesize as a selective trait. Metallothionein and ferritin bind the two most common forms of arsenic: arsenite and arsenate, respectively. Arsenate reductase catalyzes the reduction of arsenate to arsenite, allowing for the ultimate sequestration of the toxic metal to occur in the chloroplast. Transformed algae were incubated with multiple concentrations of arsenic-contaminated media and the final concentration of arsenic after 2-3 days of exposure was measured using ICP-MS to quantify uptake efficacy.

\subsubsection*{Aboa - Finland}
\paragraph{The Lac Case - Utilization of laccases for pharmaceutical waste detoxification}\mbox{} \\
Pharmaceutical waste is one of the most deleterious pollutants in the Baltic Sea. Especially the non-steroidal anti-inflammatory drug diclofenac is causing severe harm to this delicate ecosystem. The current removal efficiency of diclofenac is only 27\% at our local wastewater treatment plant. The project objective was to contribute to the development of a microbial wastewater treatment system for the detoxification of this compound. The approach was to overexpress and extract three heterologous laccases, specific enzymes which are capable of catalyzing the conversion of diclofenac into less harmful derivatives, in engineered E. coli. We were able to successfully produce and purify CotA (from B. subtilis) and CueO (from E. coli), of which CotA was shown to have catalytic activity in vitro. Implementation of this work would include the expression of this laccase in photosynthetic cyanobacteria in a closed bioreactor system, integrated as a part of the wastewater purification process.
\pagebreak
\subsubsection*{Bielefeld-CeBiTec - Germany}
\paragraph{P.L.A.N.T. Plant-based Ligand Activated Noxious agent Tracker - make the invisible visible}\mbox{} \\
As an invisible threat to the environment and human life, remnants of chemical weapons from both world wars still contaminate the soil. In Germany alone, there are over 200 suspected locations. We develop a plant-based detection system for degradation products of chemical weapons that is highly specific and allows cost-efficient screening of large areas while being easy to use. Our plant indicates the presence of toxic chemicals by changing its color to red. For this, we introduce two new reporter systems called RUBY and ANTHOS, enabling the synthesis of the plant pigments betalains or anthocyanins, respectively. If the chemical is present, it is specifically bound by a receptor, which then activates a signaling cascade, resulting in the synthesis of the pigments. Both computational design and site-directed mutagenesis are used to design new receptor proteins. In the future, our plant allows the detection of further chemicals by replacing the specific receptor.

\subsubsection*{LMSU - Russia}
\paragraph{ASCEND}\mbox{} \\
The question of food supply for the long-distance flights remains still unsolved. Moreover, onboard food production must meet restricted requirements due to the curtailed resources on the spacecraft. Сyanobacteria Arthrospira platens is a perfect candidate in this case. However, it is tasteless. A long-term ASCEND project is aimed to introduce Arthrospira platensis engineered to produce any genes of interest, and flavours in particular, as a new chassis for the synthetic biology community. We have designed and tested a special optogenetic system, which will facilitate switching between different products and help optimize growth and production conditions. Blue light induces anchoring of BcLOV4 protein in the plasma membrane and maintains culture growth, whereas far-red light induces the production of genes of interests by activating the BphP1 light-sensitive protein and forcing it to inactivate QPAS1-Gal4 repressor. This year genetic constructions were trialed in E.coli with YFP as a test gene.

\subsubsection*{Linkoping\_Sweden - Sweden} 
\paragraph{CyaSalt - A novel synthetic biology solution to the global freshwater crisis} \mbox{} \\
The world population is consistently growing and integrated agriculture is expanding consequently. As a result, the global need for freshwater is greater than ever and it continues to increase. Accordingly, the world is facing a freshwater crisis that is vastly affecting the agricultural industry in all parts of the world. CyaSalt is an innovative approach to solve this crisis. The aim of the project is to desalinate seawater in an environmentally friendly way using modified phototrophic organisms. These organisms will utilize sunlight to activate the inward-directed chloride pump, Halorhodopsin, that imports chloride ions. Sodium ions will enter via the ion channel MscL. Thereafter, the modified organisms are separated from the desalinated water by a cellulose filter. The organisms bind to the filter via a carbohydrate-binding domain on their surfaces, resulting in desalinated water free from modified bacteria. Hence, CyaSalt provides a sustainable and economically accessible freshwater source for agricultural use.
\pagebreak
\subsubsection*{MADRID\_UCM - Spain}
\paragraph{4C\_Fuels: Cyanobacterial Cyclic Carbon Capture (for sustainable bioFuel production)}\mbox{} \\
We will use cyanobacteria as living catalysts for light-driven direct carbon dioxide conversion to valuable products, upgrading the conventional biomass-based biorefineries. To do so, we will engineer the newly discovered Synechococcus PCC11801. A robust fast-growing cyanobacteria for direct sun to chemicals production. Our goal is to test the potential of photosynthetical chemical manufacturing producing n-butanol; an ideal biofuel and comodity chemical. We are implementing an artificial n-butanol biosynthetic pathway, an synthetic pathway for enhanced carbon fixation towards acetyl-CoA as central metabollite and genetic modifications for enhanced solvent tolerance. In addition we will explore cyanobacterial encapsulation in nano-structured biohybrid materials, while performing an insight into the requirements for the industrial scale-up of photobiocatalytic technology. Likewise, we will develop tools for easing cyanobacteria genetic engineering. We will develop a software for neutral integration sites identification. Also a recombination-based system for the easy generation of unmarked mutants will be developed.

\subsubsection*{Marburg - Germany}
\paragraph{OpenPlast - Establishing cell-free systems from chloroplasts as rapid prototyping platforms for plant SynBio}\mbox{} \\
Climate change is threatening many of the crops we rely on. To ensure stable food supply, engineered crops will play a major role in our future agriculture, but crop development currently takes about a decade. In our project OpenPlast, we develop cell free systems (CFS) from chloroplasts of different plants, including various crops. Showing that they can be employed as prototyping platforms to characterize genetic constructs, these systems drastically reduce testing times. We use a machine learning guided approach to optimize reaction mixture composition and create a collection of GoldenGate based chloroplast parts to be characterized in our CFS. This toolbox includes regulatory elements for chloroplasts of plants so far heavily underrepresented in the registry. After successful chloroplast transformation, we want to show that data generated in our systems is comparable to in vivo data, proving that our systems can efficiently be used as prototyping platforms for plant SynBio.

\subsubsection*{MiamiU\_OH - United States}
\paragraph{CROP: Creating RuBP Optimized Photosynthesis}\mbox{} \\
Global agricultural productivity is projected to not meet the needs of increasing populations developing higher standards of living. On a cellular level, crop yield is limited by the inefficiency of photosynthesis. Our project aims to improve this efficiency by implementing an alternative RuBP regeneration portion of the Calvin-Benson-Bassham (CBB) cycle. Two alternative pathways, which use enzymes from other reactions that act on common metabolites used in the CBB cycle, were explored first via computational modeling. Impacts on growth and reaction fluxes in silico assessed the validity of these pathways in creating a more robust photosynthetic cycle. One of these pathways which overexpresses the native enzyme transaldolase, was also assessed in vivo. Ultimately, we showed the validity of two alternative pathways in allowing a more efficient regeneration stage of the CBB Cycle. These pathways could eventually be implemented into higher plants to allow more robust cycling and therefore higher crop yield.
\pagebreak
\subsubsection*{Sorbonne\_U\_Paris - France}
\paragraph{Chlamy'n Space}\mbox{} \\
Acute or chronic exposure to ionizing radiation (UVC, gamma rays, X rays…) leads to the formation of reactive oxygen species (ROS) which affect the genome and the proteome of cells. Thus, although photosynthetic microorganisms such as Chlamydomonas reinhardtii constitute one of the main hopes for developing Bioregenerative Life-Support System during long-term space travel, their use is called into question by the decrease of efficiency of the photochemistry and by the growth arrest caused by ROS. Our project aims to make Chlamydomonas reinhardtii produce a peptide complexing with the Mn2+ ion inspired by a metabolite found in the radioresistant organism Deinococcus radiodurans and acting as an antixoidant. This study requires to demonstrate a decrease in ROS within the cell during production of the peptide and verifying the growth of microalgae cultivated in minimum medium (photosynthesis dependent growth).

\subsubsection*{Toulouse\_INSA-UPS - France}
\paragraph{Elixio, a synthetic microbial consortium for sustainable violet fragrances}\mbox{} \\
Perfumes influence perception in our daily life. Behind flowers and chic clichés, perfume reality is not so glamorous as most are issued from non-sustainable processes. This is especially true for scents impossible to extract from the so-called “mute flowers” like the violet. Our Elixio project aims to demonstrate that valuable fragrances could be easily recreated using synthetic biology, even by a small team of students. We designed a synthetic consortium involving engineered cyanobacterium and yeast and allowing a sustainable production of the violet scent molecules from atmospheric CO2. Over the summer, we successfully engineered both strains to conditionally express all the enzymes necessary to recreate the violet fragrance. Moreover, we demonstrated the production of ionones by our yeast which actually smells like violet! The Elixio project has already drawn attention from the industry and we are definitely proud of the new openings created between iGEM and the world of perfume.

\subsubsection*{Victoria Wellington - New Zealand}
\paragraph{Tropane alkaloid biosynthesis in prokaryotes}\mbox{} \\
Tropane alkaloids are plant secondary metabolites and include important medicinal compounds. Most applications are related to neurochemistry and range from the treatment of neuromuscular disorders, including Parkinson’s disease, to the use as stimulants. There is increasing need for large-scale, climate-independent, and local production of tropane alkaloids as precursors for medicinal drugs. Our goal is to remedy the impact of world crises on the cultivation and exportation of these drugs. To this avail, we aim to develop a biosynthetic route for a tropane alkaloid intermediate, tropine, in Escherichia coli and the cyanobacterium Synechococcus elongatus. To the best of our knowledge this would be the first production of tropine in a prokaryotic organism and could provide an effective and cheap alternative to current tropine production methods.

% =====================================================================
% ============================== SECTION ==============================
% =====================================================================
\pagebreak
\section{Phototrophic Chassis in iGEM}
Synthetic biology relies on the use of chassis organisms. They are re-designed, so that they gain functions the organisms wouldn’t have in a normal environment. To use an organism as a chassis in synthetic biology, it needs to be well described and easy to cultivate and have high growth rates. E. coli fulfills all of the criteria and therefore is one of the most used chassis organisms in synthetic biology. In contrast to that, synthetic biology of plants, algae and cyanobacteria is currently very rarely applied. Synthetic biologists all over the world are currently researching tools for phototrophic synbio, e.g. logic circuits, transformation methods or metabolic engineering.
\newline\newline
This situation is also reflected in iGEM, where the large majority of teams uses bacteria, mainly E. coli, as their chassis and phototrophic organisms are a quite small niche.
Due to the limited time available to most iGEM teams, it simply does not make sense for the majority of teams to invest this already very confined amount of time working on phototrophs.
\newline\newline
In 2016, the special prize “Best Advancement in Plant Synthetic Biology” was awarded for the first time to the overgrad team Cambridge-JIC, which invented a toolbox for chloroplast transformation of the green algae \textit{Chlamydomonas reinhardtii}. The first undergrad team, which won the special prize was SCAU-China. They expressed astaxanthin in rice. In 2017 this special prize was renamed to “Best Plant Synthetic Biology” and awarded every following year to undergrad and overgrad teams working with phototrophic chassis.

% =====================================================================
% ============================== SECTION ==============================
% =====================================================================

\section{Project Ideation}
As an iGEM Team it can be hard to decide for a certain project in the beginning- there are so many things that want to be thought about, among other things the organism of choice. It is likely that your team has been thinking about a project using phototrophic organisms like algae, cyanobacteria or higher plants if you are reading this. We, being iGEM phototrophic-organism-project alumnis, will try to help you decide whether a project using phototrophic organisms is right for your team.
\newline\newline
There is one thing independent from the phototrophic organism of choice: gaining the necessary knowledge. For most teams coming from a bacterial background working with phototrophic organisms - especially plants - seems deterrent at first, giving their lack of knowledge. To help with the first steps we created this handbook. Beside that it is really helpful to have local know-how like a PI working with the corresponding phototrophic organisms or a university working group.
\newline\newline
Otherwise the different phototrophic organisms require different considerations and preparations.
\newline\newline
Let's start with talking about higher plants. First of all: plants are great! Depending on your project idea you already might have stumbled upon the immense potential of plant synbio applications. 
There are the obvious improvements for agricultural purposes regarding stress tolerance, increasing their yield or adding pest resistances. 
Beside that there are also biofuel applications, implementation of new metabolite pathways and expressing therapeutics which makes plants as organisms quite versatile.
\newline\newline
When working with higher plants there are a few things to consider beforehand. The main point is time consumption since you can't just throw some cells from a cryo-stock into medium and work with them the next day as it is possible with bacteria. Acquiring plants, both from a university-intern greenhouse or growing them themselves requires a lot of time and planning ahead: when do we need how many plants? 
\newline\newline
Another point is the transformation of plant cells. We have a whole chapter dedicated to that but the short summary is that it also takes time and is a relatively big effort. The standard method for transient transformation is agroinfiltration: beforehand transformed agrobacteria are injected into the leaves of the plant, infecting the plant-cells and transforming the tissue. Stable transformation takes even more time needing several generations of plants. Fortunately  for a proof of concept transient transformation is usually sufficient.
\newline\newline
A relevant point, especially for us iGEMers is the availability of Biobricks and parts that can be used in our constructs. Plant synbio being a rather new synbio field is still lacking a lot of well-characterized and interchangeable parts. The amount of parts e.g. on \\ http://parts.igem.org/Collections/Plants is growing bigger every year but compared to other part-collections still relatively small. Additionally compatibility issues like codon optimization, genetic instability and regulatory incompatibility should be considered when creating constructs.
\newline\newline
The final point regarding the work in the lab is about evaluation. It should be considered that evaluating and analysis methods can differ from those used for microorganisms. 
\newline\newline
Depending on your project and your region it should be considered that GMO-plants outside the laboratory can be a tricky thing to justify in the human practice sub-category of iGEM and real world application will not happen in the near future. 

% =====================================================================
% ============================== SECTION ==============================
% =====================================================================

\section{Planning \& Getting Started}
In this chapter you will find further information and tips on how to organise a project with a phototrophic organism. 
\newline\newline
The first step is to acquire the strain of choice. This can be more time consuming than initially planned. Especially for experiments with higher plants it is essential to know when how many plants of what age are needed. Obviously this needs to be known a few weeks ahead. 
\newline\newline
There are generally two ways of getting hands on plants: ordering the plants from the \\(university-) greenhouse or growing them themselves.
Ordering the plants needs good communication about the age and time when they are needed. Another consideration might be the type of soil the plants are raised and how the plants are treated once delivered.
\noindent
When raising your own plants it is important to standardize the growth conditions for every plant to get an even population. The main variables are light exposure and source as well as watering and fertilizer. 
Last but not least there are different cultivation methods- in soil, in agar and in liquid, there is a separate chapter for growth and cultivation methods.
\newline\newline
If your laboratory doesn't have equipment for growing and handling plants or other phototrophic organisms, e.g. in photobioreactors, those need to be organized ahead of time, probably bought. There are online grow-shops in every region that can be used to order equipment like grow-lamps and more.
\newline\newline
When evaluating and analysing results it should be considered that protocols or necessary laboratory-devices might differ from organism to organism and may need reservation. Also asking for advice is always recommended.

