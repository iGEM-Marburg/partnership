\section{Measurement}
\epigraph{The following chapter was written by Michael Burgis}{\textit{iGEM Marburg 2021}}
\noindent
With this chapter we would like to describe good practice for the test phase during an iGEM project and what aspects need to be especially taken into consideration when working with phototrophic organisms.
\\ \\
In synthetic biology, measurement refers to the process of testing and characterising a system or device. Good measurement ensures that the data you produce is relevant, comparable, accurate, reliable, and reproducible.
To ensure the collected data to fulfill all of these requirements a lot of planning and supporting experiments need to be designed.
\\ \\
Most of the data generated by the iGEM teams tends to be collected during the end of the project. Therefore it is of high priority to work out potential problems and the general design of the measurements that are to be conducted as soon as possible. As a useful resource, iGEM itself already offers a helpful question catalog about which aspects you may consider before planning your experiments. This catalog can be found following this website \\ \href{https://2021.igem.org/Engineering/Test}{https://2021.igem.org/Engineering/Test} under the section \textbf{Question to consider for wet lab projects}.
In addition, it is very important to plan how to address the identified problems.
\\ \\
While the aforementioned link also includes resources like the description to use iGEM-provided standards or measurement protocols, they are mainly targeted to the well established lab chassis \textit{E.coli} and are only partially suited for phototrophs making the generation of quantitative data comparably harder.
\\ \\
Chlorophyll and other autofluorescent molecules present in phototrophs can present quite some troubles considering optical density and fluorescence measurements. 
For OD measurements it is important to know the absorption spectrum of your organism of choice and then choose the wavelength accordingly. The measurement of microalgal and bacterial cultures using a photometer or a plate reader is generally done through measuring the turbidity of the cell suspension. This is done indirectly by evaluating how much of the emitted laser is being deflected by the cell suspension. As a consequence, the wavelength should be chosen where the absorption of the organism is relatively negligible. For cyanobacteria and algal cells, 730nm is the most common wavelength setting \parencite{Crozet2018} \parencite{Niederholtmeyer2010} \parencite{Yu2015} \parencite{Wlodarczyk2020} \parencite{Kachel2020}.
\\ \\
Fluorescent reporters are well established in the field of molecular biology. They can be used in plants, algaes and cyanobacteria alike and can be used to characterize genetic parts. Similar to OD measurement, fluorescence measurements are affected by autofluorescence. Due to major peaks of autofluorescence emission in the blue to green \parencite{Rasala2013} and the far red range \parencite{Kavna2009}, we mainly advise to use reporters that emit in the yellow to red range.
The website \href{https://www.fpbase.org/}{fpbase} is a good tool to find a suitable reporter for your project. They offer a database of 784 fluorescent proteins and for most of these proteins accurate excitation and emission spectra are available.


\begin{table} [!htpb]
\centering
\resizebox{\linewidth}{!}{%
\begin{tabular}{|lllllllll|} 
\hline
\begin{tabular}[c]{@{}l@{}}\textbf{Fluorescent}\\\textbf{protein}\end{tabular} & \textbf{Excitation λ}~ & \textbf{Emission λ}   & \begin{tabular}[c]{@{}l@{}}\textbf{Extinction}\\\textbf{coefficient}\end{tabular} & \begin{tabular}[c]{@{}l@{}}\textbf{Quantum}\\\textbf{Yield}\end{tabular} & \textbf{Brightness}     & \textbf{pKa}          & \begin{tabular}[c]{@{}l@{}}\textbf{Maturation}\\\textbf{time (min)}\end{tabular} & \textbf{Lifetime (ns)}  \\ 
\hline
\textbf{LanYFP}                                                                & \textbf{513}           & \textbf{524}          & \textbf{150.000}                                                                  & \textbf{\textbf{0.95}}                                                   & \textbf{\textbf{142.5}} & \textbf{\textbf{3.4}} & -                                                                                & -                       \\ 
\hline
\textbf{mScarlet}                                                              & \textbf{569}           & \textbf{\textbf{594}} & \textbf{\textbf{100.000}}                                                         & \textbf{\textbf{0.7}}                                                    & \textbf{\textbf{70}}    & \textbf{\textbf{5.3}} & \textbf{\textbf{174}}                                                            & \textbf{\textbf{3.9}}   \\
\hline
\end{tabular}
}
\end{table}
\FloatBarrier
\noindent
A possibility to avoid this background fluorescence is by switching to bioluminescent reporters. These enzymes can catalyze reactions for specific substrates and in turn produce a light signal that can be detected by an analytical device of choice. The signal from this reaction is generally high and shows low background making it a suitable reporter for phototrophs. But due to its mechanism it is only possible to perform end point measurements.
\\ \\
Besides these usual reporter genes, we can recommend a recently published reporter system based on plant pigments called RUBY \parencite{He2020}. It is based on the betalains that can be used for quantitative measurements using photometric measurements. This system consists of the three enzymes needed to convert tyrosine to betalains. In RUBY, they are combined in one single open reading frame and separated by 2A peptides, which can later separate the proteins by self-cleavage. So, the three enzymes are transcribed and translated together. In the end, the three individual enzymes are released and are completely functional. 
\\ \\
Plate readers and other equipment using photometrics in order to measure samples are not usable for the generation of comparable data. Every lab has different equipment with different settings, and measurements of fluorescence or absorbance from this equipment are often reported using arbitrary units (AU). These AU values from different labs cannot be directly compared, which hinders reproducibility and can discourage others from building on your work and/or using your systems.
\\ \\
Measurement standards can be used to calibrate equipment and convert arbitrary units to absolute units, which are comparable. In the past the iGEM foundation usually included these standards in their distribution package. These standards included fluorescein, which can be used to convert arbitrary units of fluorescence to values equivalent to a molecule of fluorescein \parencite{Beal2018}. Also, a standard for the conversion of OD to special silica beads was included in order to better quantify the amount of \textit{E.coli} cells in a sample \parencite{Beal2020}. While the fluorescein normalization can be done for phototrophs as well, the silica bead normalization depends on the size of the organism. Therefore it should be possible to convert arbitrary OD measurements for phototrophs using silica beads with the right size. Unfortunately, no standard for luminescent reporters has been established making it impossible to quantitatively compare luminescent data across laboratories.




\section{Normalization}
Although there are a lot of external factors that can be standardized for more reproducible plant growth and data generation, some biological processes simply cannot be standardized.
\\ \\
In order to generate more reproducible data using phototrophs we propose to use a dual reporter system. The idea is to always include a gene cassette in every construct  that stays the same. This allows us to statistically normalize to it and eliminate variation between measurements. This way it is possible to produce more comparable data across different experiments. In literature this system was successfully used in protoplasts of \textit{Arabidopsis thaliana} and \textit{Sorghum bicolor} and compared to actual stable transformants \parencite{Schaumberg2015}.
\\ \\
This system can be used in conjunction with fluorescent reporters as well as luminescent reporters. For fluorescence we advise to use the two proteins we mentioned in the last chapter. Their difference in the emission maxima makes it possible to use the two peptides and measure them in one run without too much background signal by choosing the excitation wavelength accordingly.
\\ \\
For luminescence we advise to use commercial dual luciferase kits. These kits usually lyse the cell material and subsequently measure the reporters in the same reaction mixture. This is achieved by quenching the signal of the first luciferase. On the one hand these systems are very sensitive and exhibit low background signal, but on the other hand these systems are very costly and are not suited for time series data generation.

