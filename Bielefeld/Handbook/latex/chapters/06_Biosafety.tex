\section{Cartagena Protocol and Biosafety Resources}
\epigraph{The following chapter was written by Lidia Bobrovnikova \& Ilya Rubinstein}{\textit{iGEM LMSU 2021}}
\noindent
Living modified organism (LMO) is literally any organism whose genome was changed in any way. There is a great variety of higher plants which appear to be LMO, e.g. maize, cotton, soya beans, etc. \\
The concept of biosafety encompasses a range of measures, policies and procedures for minimizing potential risks that biotechnology may pose to the environment and human health. Establishing credible and effective safeguards for GMOs is critical for maximizing the benefits of biotechnology while minimizing its risks. Such safeguards must be put in place now, while biotechnology is still relatively young. 

\subsection{BCH}
After the foundation of Cartagena protocol more people became more aware of biosafety while handling organisms. This caused the development of databases for further research. The most popular platform today is \href{https://bch.cbd.int/protocol/text/}{BCH}, where most biosafety protocols are published. There can also be found the geography, ancestors, toxicity, composition, history of modifications etc. of any strains and cell cultures around the globe. 

\subsection{IPPC}
\href{https://www.ippc.int}{IPCC (international plant protection convention)} is responsible for tracking the appearance of transgenic plants, preservation of the primary species, etc. 
 

\section{Plant Biosafety Levels}
There are four biosafety levels and associated practices for plant research. Only BSL-1P and BSL-2P research is conducted at the UW. There are no BL3-P or BL4-P experiments at the UW. 
\subsection{BSL-1P}
BSL-1P is recommended for transgenic plants that are not noxious weeds, are not easily disseminated and are not detrimental to the environment.\\ 
Requirements at BSL-1P include:
\begin{itemize}
    \item Access to the laboratory and greenhouse is limited or restricted when experiments are in progress.
    \item Prior to entering the greenhouse, personnel are required to read and follow instructions on BSL-1P greenhouse practices and procedures.
    \item All procedures must be performed in accordance with accepted greenhouse practices appropriate to the experimental organism.
    \item Records will be kept of experiments currently in progress in the greenhouse facility.
    \item Render experimental organisms biologically inactive by appropriate methods before disposal.
    \item A program shall be implemented to control undesired species (e.g., weed, rodent, or arthropod pests and pathogens) by methods appropriate to the organisms and in accordance with applicable state and federal laws.
    \item House arthropods and other motile macroorganisms in appropriate cages. If macroorganisms (e.g., flying arthropods or nematodes) are released within the greenhouse, precautions must be taken to minimize escape from the greenhouse facility.
    \item The greenhouse floor may be composed of gravel or other porous material. Impervious (e.g., concrete) walkways are recommended.
    \item Windows and other openings in the walls and roof of the laboratory and greenhouse facility may be open for ventilation as needed for proper operation and do not require any special barrier to contain or exclude pollen, microorganisms, or small flying animals (e.g., arthropods and birds). Screens are recommended.
    \item Laboratories and greenhouses must be locked when unoccupied. All agents must be secured against accidental exposure, unauthorized use, and theft. All recombinant nucleic acids must be stored in locked containers.
\end{itemize}
\subsection{BSL-2P}
BSL-2P is recommended for transgenic plants that are noxious weeds or can interbreed with noxious weeds, transgenic plants that contain the genome of an non-exotic infectious agent, and transgenic plants or plant pathogens that may have a detrimental impact to the environment.
The following are required when working at BSL-2P:
\begin{itemize}
    \item A program to control undesired species (e.g., weed, rodent, or arthropod pests and pathogens) by methods appropriate to the organisms and in accordance with applicable state and Federal laws.
    \item A greenhouse floor composed of an impervious material. Concrete is recommended, but gravel or other porous material under benches is acceptable unless propagates of experimental organisms are readily disseminated through soil. Soil beds are acceptable unless propagates of experimental organisms are readily disseminated through soil.
    \item Materials containing experimental microorganisms must be transferred in a closed , leak proof container.
    \item An autoclave must be available for the treatment of contaminated plant material including soil.
    \item If intake fans are used, measures shall be taken to minimize the ingress of arthropods. Louvers or fans shall be constructed such that they can only be opened when the fan is in operation.
    \item Greenhouse containment requirements may be satisfied by using a growth chamber or growth room within a building provided that the external physical structure limits access and escape of microorganisms and macroorganisms in a manner that satisfies the intent of the foregoing clauses
    \item Laboratories and greenhouses must be locked when unoccupied.
    \item All agents must be secured against accidental exposure, unauthorized use, and theft.
    \item All recombinant nucleic acids and BSL-2P agents must be stored in locked containers.
    \item All material in the open bay or common use areas must be secured when not in use.
\end{itemize}
\subsection{Plant Containment and Handling} 
Plant containment can be physical or biological. Physical containment can include a plant growth chamber or a greenhouse, whereas biological containment refers to removal or inactivation of plant reproductive structures (pollen and seed). Methods to contain or handle plants may include:
\begin{itemize}
    \item Facilities such as research labs, growth rooms, or greenhouses
    \item Procedures, practices, and personal protective equipment (PPE)
    \item Transportation and storage
\end{itemize}
For transgenic plant research, standard operating procedures (SOPs) that describe methods to contain plants, associated organisms and waste are required.
\href{https://www.ehs.washington.edu/biological/biological-research-approval}{Biological Use Authorization (BUA)} is required for all plant research involving recombinant or synthetic nucleic acids (DNA/RNA).
Regulations governing plant research depend on the type of research, and whether plants, plant pathogens, plant-associated arthropods, noxious weeds, invasive plants, or other biological agents are used.
\href{http://www.aphis.usda.gov/aphis/resources/permits}{APHIS permits} may be required.

\subsection{Field Work} 
Field work with transgenic plants requires permits from APHIS prior to the start of work. If you have any plans to do field work with genetically engineered plants or associated organisms, inquire about permits first.
\subsection{Biohazardous Plant Waste}
Plant waste, including all transgenic plants, seeds, spores, plant debris and soil materials, and any plants exposed to plant pathogens are considered biohazardous waste and must be inactivated prior to disposal. Refer to \href{https://www.ehs.washington.edu/biological/biohazardous-waste}{Biohazardous Waste} for more information on methods of decontamination and disposal.

\section{Edible Microalgae}
Microalgae have a significant potential as a food source. Their biomass contains a great amount of essential for human compounds (unsaturated fatty acids, amino acids, vitamins, flavonoids etc.). Moreover, production of microalgae is economically sustainable. However, today the price for microalgae biomass suitable for nutrition is quite high, though it can be curtailed if production is moved to a greater scale. 
Nowadays, the most popular species of microalgae for food are \textit{Spirulina} (\textit{Arthrospira sp.}) and \textit{Chlorella sp.} Both these organisms are used today as a food additive in animal and human diet. They produce all essential amino acids, most vitamins and essential unsaturated fatty acids. Interesting to note, that their biomass is richer in protein than soybeans. \\
Overall, microalgae contain a great variety of organic compounds: volatile compounds, sterols, proteins, all classes of lipids and polysaccharides etc. However, some compounds, especially volatile and phenolic compounds, can be hazardous \parencite{Andrade2018} \parencite{Torres-Tiji2020}.

\section{Toxins of Microalgae and Cyanobacteria}
Some microalgae are able to produce by-products or as “waste” molecules some toxic compounds. The most popular classes of such toxins are neurotoxins and hepatotoxins. 

\subsection{Hepatotoxins (Microcystins, Nodularines)}
These molecules have a particular for this class of toxins mechanism of action and biosynthesis. They can cause such symptoms as high temperature, diarrhea and indisposition. These molecules are not commonly produced by cyanobacteria and their synthesis is initiated with the separation of cells in the colony. This can be avoided when using heating and MV of cultures.

\subsection{Neurotoxins (Anatoxins, Saxitoxins)}
This class is characterised by numerous mechanisms of action and different molecule stability. \\ 
Anatoxins are less hazardous for human, however, may cause lethality. These molecules are quite stable, but can be broken down by cyanobacteria themselves after a long period of storing of culture in non-sterile conditions. 
Saxitoxins are probably ones of the most hazardous of the known substances. Their action is slighter than in anatoxins, but their stability is much greater and they can be synthesized \textit{in vitro}.  These toxins are produced by dinoflagellates and diatoms and there are quite a lot of cases of poisoning (red tides — harmful algal blooms) \parencite{Yunes2019} \parencite{Sivonen2009}. 

\section{Catastrophes caused by Plants, Algae and Cyanobacteria}
\subsection{Water Blooms}
Algae blooms are an increase of algae number in the water. This phenomena can be seasonal or spontaneous. Seasonal blooms are often caused by prolonging the day, increasing the temperature and oxygen concentration near the water layer. These blooms are not dangerous for ecosystems and do not cause catastrophes due to the variation of different species with different toxins in the water. \\
In contrast, spontaneous blooms are more noxious for ecosystems. There are a few reasons for that, but the most popular explanation is eutrophication — an excess of  biogenic compounds in the water. Eutrophication can be both natural and caused by humans. A rapid increase of nutrients of water causes a massive propagation of algae and, subsequently, a critical accumulation of dissolved oxygen and growth of toxic compounds concentration produced by algae. Finally, the complex of aforementioned reasons is followed by an extinction of water animals. \\
Water blooms can be circumvented by following the rules of national and specific communities and making an analysis of water conditions. 

\subsection{Invasive Plants}
It is common knowledge that many plants are able to adapt to new conditions and repress the indigenous species by bringing changes into ecosystem hierarchy. \\
One such example is \textit{Elodea canadensis}. Initially this species comes from North America, but today it is spread to Europe, Asia and Africa. This species has a very high growth rate and ability to survive in rapidly changing conditions. 
How can we prevent the contamination of ecosystems with invasive species? Thinking as pessimists — there is no way. This is a natural process of spreading species. A new challenge for scientists!
