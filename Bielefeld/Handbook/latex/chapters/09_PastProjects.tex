\section{Algae}

\textbf{\uppercase{Humboldt\_Berlin}} 
\FloatBarrier
\begin{table}[h]
\begin{tabular}{lllll}
\textbf{Location:} & Germany & \multicolumn{1}{|l}{} & \textbf{Track:}   & Environment \\
\textbf{Region:}   & Europe   & \multicolumn{1}{|l}{} & \textbf{Section:} & Overgraduate \\
\textbf{Year:}     & 2019   & \multicolumn{1}{|l}{} & \textbf{Awards:}  & Gold Medal
\end{tabular}
\end{table} 
\FloatBarrier
\noindent\textbf{Chlamylicious - Establishing Chlamy at iGEM while degrading plastic} \vspace{.2cm}\\ 
Chlamydomonas reinhardtii is a unicellular algae with promising prospects for synthetic biology. Its ability to grow photoautotrophically makes it an ideal chassis to tackle a variety of problems in an environmentally friendly way. Our goal is to adress the worldwide problem of plastic pollution by creating a catalogue of genetic parts for C. reinhardtii that can enable the algae to degrade PET plastic. By combining different functional genetic parts we plan to address the problem from multiple perspectives. To do so, we are designing and building a reproducible low-budget cultivation setup which will aid us and others in the process of collecting data of algal growth under the influence of transgenic constructs and other parameters.latile tool for dealing with a complex problem such as plastic pollution from different perspectives. 
\vspace{2cm} $ $
\pagebreak

\noindent\textbf{\uppercase{Cambridge-JIC}} 
\FloatBarrier
\begin{table}[h]
\begin{tabular}{lllll}
\textbf{Location:} & United Kingdom & \multicolumn{1}{|l}{} & \textbf{Track:}   & Foundational Advance \\
\textbf{Region:}   & Europe   & \multicolumn{1}{|l}{} & \textbf{Section:} & Overgraduate \\
\textbf{Year:}     & 2016   & \multicolumn{1}{|l}{} & \textbf{Awards:}  & Gold Medal
\end{tabular}
\end{table} 
\FloatBarrier
\noindent\textbf{InstaChlam - a toolkit for chloroplast transformation} \vspace{.2cm}\\
As the factory floor of the plant cell, the chloroplast can be engineered to produce many important compounds, such as biofuels and vaccine antigens, with yields approximately 50X greater than the rest of the cell. However, little of this potential has been exploited, in the absence of a time-efficient chloroplast transformation protocol. Using the alga Chlamydomonas reinhardtii as our chassis, our transformation toolbox aims to shift the focus of plant engineering, by reducing the time needed for a homoplasmic chloroplast transformation from months to 1-2 weeks. We have created a library of Chlamydomonas-optimised parts in the Phytobrick standard, with a view to expressing Cas9 in the Chlamydomonas chloroplast for the first time. We hope to use it to propagate genetic modifications among all copies of the chloroplast genome, within a single generation. We will also develop a low-cost, open source Chlamydomonas growth facility and gene gun to complement our protocol. 
\vspace{2cm}

\noindent\textbf{\uppercase{FAFU-CHINA}} 
\FloatBarrier
\begin{table}[h]
\begin{tabular}{lp{2.5cm}llll}
\textbf{Location:} & China & \multicolumn{1}{|l}{} & \textbf{Track:}   & Environment \\
\textbf{Region:}   & Asia   & \multicolumn{1}{|l}{} & \textbf{Section:} & Undergraduate \\
\textbf{Year:}     & 2016   & \multicolumn{1}{|l}{} & \textbf{Awards:}  & Gold Medal
\end{tabular}
\end{table} 
\FloatBarrier
\noindent\textbf{Cry For Mosquito} \vspace{.2cm}\\
In 2016, FAFU-CHINA will attach the effective protoxin gene, which isolated from Bacillus thuringiensis contained the characteristic of high efficient mosquito control, to Chlamydomonas reinhardtii as the chassis organism for cloning. However, there are many problems in the practical application by using Bacillus thuringiensis, such as the bacterial pollution of waters, or the poor timeliness which Bt. strains cannot colonize in the water. Our team use the pertinent literature as the basis for selecting the appropriate biological chassis, combinating the toxic protein, which in order to increase the effect of killing mosquito larvae and reduce the Bt. toxin tolerance. on the basis of Chlamydomonas reinhardtii expression system to optimize gene,enhance expression of results, and reconstruct the engineered bacteria in natural environment, we can settle the problems mentioned above. 
\vspace{2cm} $ $
\pagebreak

\noindent\textbf{\uppercase{SZU-China}} 
\FloatBarrier
\begin{table}[h]
\begin{tabular}{lp{2.5cm}llll}
\textbf{Location:} & China & \multicolumn{1}{|l}{} & \textbf{Track:}   & Energy \\
\textbf{Region:}   & Asia   & \multicolumn{1}{|l}{} & \textbf{Section:} & Undergraduate \\
\textbf{Year:}     & 2016   & \multicolumn{1}{|l}{} & \textbf{Awards:}  & Gold Medal
\end{tabular}
\end{table} 
\FloatBarrier
\noindent\textbf{Light Hygician} \vspace{.2cm}\\
Hydrogen energy, is of great potential in the future with its zero-emission and high-efficiency. However, the fact that few efficient and environment-friendly methods for hydrogen production constrains its application. Therefore, our team develope a biological production way, using the green algae - Chlamydomonas reinhardtii. Since the hydrogenase activity will be inhibited in absence of oxygen and the algae can’t stop photosynthesis forever, we design a switch altering between 2 states in which light wavelength serves as extraneous inducible factor. In specific alternation, we utilize miRNA targeting the expression of key protein in photosynthesis, so we can select the hydrogen-production switch by regulating miRNA. In our design, we utilize Yeast-Two-Hybrid system and light-mediated fusion protein constructing a gene circuit where microRNA can regulate the specific downstream protein expression, and finally keep algae producing H2. In this way, the blue light switch regulate chlamydomonas producing intemittent hydrogen efficienctly, acting as blue-flame bubbling.
\vspace{2cm}

\noindent\textbf{\uppercase{Linkoping\_Sweden}} 
\FloatBarrier
\begin{table}[h]
\begin{tabular}{lp{2.5cm}llll}
\textbf{Location:} & Sweden & \multicolumn{1}{|l}{} & \textbf{Track:}   & Environment \\
\textbf{Region:}   & Europe   & \multicolumn{1}{|l}{} & \textbf{Section:} & Overgraduate \\
\textbf{Year:}     & 2016   & \multicolumn{1}{|l}{} & \textbf{Awards:}  & Gold Medal
\end{tabular}
\end{table} 
\FloatBarrier
\noindent\textbf{A CRSIPR case for biofuel} \vspace{.2cm}\\
This year LiU iGEM will be a part of the search for alternative energy sources, this as a result of global warming due to excessive use of fossil fuels. For this we will use CRISPR/Cas9 in unicellular model algae Chlamydomonas reinhardtii, which has shown great potential for production of biofuels. Previous research has attempted to modify algae to promote the lipid synthesis thereby optimizing them for biofuel production. In this project we want to create new Biobricks consisting of inducible promoters to couple with a CRISPR/Cas9-system in C. reinhardtii. The reason for the inducible promoters is to avoid complications such as toxicity of a constitutively active Cas9. By causing cultures of algae to undergo genetic modification in response to high intensity light we believe we can solve this problem. With this method genes can be targeted in the model algae in order to regulate the lipid synthesis. \\ \\
\vspace{2cm}  $ $
\pagebreak

\noindent\textbf{\uppercase{USP\_UNIFESP-Brazil}} 
\FloatBarrier
\begin{table}[h]
\begin{tabular}{lp{2.5cm}llll}
\textbf{Location:} & Brazil & \multicolumn{1}{|l}{} & \textbf{Track:} & Manufacturing \\
\textbf{Region:} & Latin America   & \multicolumn{1}{|l}{} & \textbf{Section:} & Overgraduate \\
\textbf{Year:}     & 2016   & \multicolumn{1}{|l}{} & \textbf{Awards:}  & Gold Medal
\end{tabular}
\end{table} 
\FloatBarrier
\noindent\textbf{AlgAranha} \vspace{.2cm}\\
The objective of this project is to produce a biomaterial for immobilizing proteins initially directed to application on burns, using immobilized enzibiotics. The term "enzibiotics" refers to the junction of the words "enzyme" and "antibiotic," this is enzymes exhibiting antimicrobian activity. For immobilizing these biomolecules will be used gene recombination techniques, adding the polymerization domains in the enzibiotic molecule, compatible with the spider silk proteins. Both will be produced in recombinant microalgae by nuclear transformation of model microorganism Chlamydomonas reinhardtii. The project will be executed by group of undergraduates and graduate students in the context of iGEM. It is expected to achieve spider silk fiber production and its initial characterization for antimicrobial activity and mechanical properties, as well as the productivity evaluation in the proposed system. From these results, we can evaluate the application of this immobilizer in other biotechnological applications such as biotransformation, biosensors, biomaterials and textile industry.
\vspace{2cm}

\textbf{\uppercase{UConn}}
\FloatBarrier
\begin{table}[h]
\begin{tabular}{lp{2.5cm}llll}
\textbf{Location:} & United States & \multicolumn{1}{|l}{} & \textbf{Track:}   & Energy \\
\textbf{Region:}   & North America   & \multicolumn{1}{|l}{} & \textbf{Section:} &  \\
\textbf{Year:}     & 2017   & \multicolumn{1}{|l}{} & \textbf{Awards:}  &
\end{tabular}
\end{table}
\FloatBarrier
\noindent	extbf{} \vspace{.2cm}\\
An Algaeneious Approach to Continuous Cultures for Biofuel Production
Biofuels are a promising, nearly carbon-neutral alternative fuel source, often derived from algal lipid production. Previous methods of fuel harvest have relied on destructive means of extraction, but we aim to upregulate the excretion of lipids, allowing for potential harvest by physical separation. Our goal is to enhance the algal lipid production and extracellular transport in Nannochloropsis oceanica, a well characterized species with high lipid content. This will be achieved by upregulating the endogenous lipid production enzymes of the cell with high expression promoters and transfecting algae with an ATP binding cassette transporter from Arabdopsis thaliana. Next steps will include developing a system to physically separate excreted lipid from the algal biomass, while maintaining a productive continuous culture.

% ========== TEMPLATE ==========
\iffalse
\textbf{\uppercase{Team\_Name}} 
\FloatBarrier
\begin{table}[h]
\begin{tabular}{lp{2.5cm}llll}
\textbf{Location:} & Germany & \multicolumn{1}{|l}{} & \textbf{Track:}   & Environment \\
\textbf{Region:}   & Europe   & \multicolumn{1}{|l}{} & \textbf{Section:} & Undergraduate \\
\textbf{Year:}     & 2019   & \multicolumn{1}{|l}{} & \textbf{Awards:}  & Gold Medal
\end{tabular}
\end{table} 
\FloatBarrier
\noindent\textbf{TITLE} \vspace{.2cm}\\
Abstract
\fi

\pagebreak
\section{Cyanobacteria}

\textbf{\uppercase{Amsterdam}} \FloatBarrier \begin{table}[h] \begin{tabular}{lp{2.5cm}llll} \textbf{Location:} & Netherlands & \multicolumn{1}{|l}{} & \textbf{Track:}   & Energy \\ \textbf{Region:}   & Europe   & \multicolumn{1}{|l}{} & \textbf{Section:} &  \\ \textbf{Year:}     & 2015   & \multicolumn{1}{|l}{} & \textbf{Awards:}  & \end{tabular} \end{table} \FloatBarrier \noindent\textbf{Synthetic Romance - Harnessing the power of Cyanobacteria to construct a sustainable consortia} \vspace{.2cm}\\  Researchers are starting to recognize that synthetic ecosystems consortia of multiple bacterial species can be used for higher yields robustness and more diverse purposes. Our goal is to tap into this potential by creating a self-sustaining bio-factory of cyanobacteria - little fellows that need only CO2 and light - and product-producing E. coli the general workhorse of the synthetic biology world. The cyanobacteria will create sugars from CO2 and sunlight which it will release and feed to E. coli as a result of our applied synthetic genetic circuits. E. coli will then be engineered to use these sugars to create a product. In our proof-of-concept bio-factory this product will be fuel. This platform however can be expanded to produce any product E. coli can make - medicine plastics commodity chemicals - as long as it is fueled by the cyanobacteria that only needs light and CO2.
\vspace{2cm}

\textbf{\uppercase{LaVerne-Leos}} \FloatBarrier \begin{table}[h] \begin{tabular}{lp{2.5cm}llll} \textbf{Location:} & United States & \multicolumn{1}{|l}{} & \textbf{Track:}   & Energy \\ \textbf{Region:}   & North America   & \multicolumn{1}{|l}{} & \textbf{Section:} &  \\ \textbf{Year:}     & 2015   & \multicolumn{1}{|l}{} & \textbf{Awards:}  & \end{tabular} \end{table} \FloatBarrier \noindent\textbf{Using Zeaxanthin and Tocopherol to protect cyanobacteria from the toxic effects of free fatty acids} \vspace{.2cm}\\ Free fatty acids are biofuel precursors.  We focused on using zeaxanthin to counter the toxic effects of increased free fatty acids in an altered Synechococcus elongatus 7942 strain. Zeaxanthin acts as an antioxidant stabilizes the membrane and is needed for electron transport chain function. To increase the concentration of zeaxanthin and its precursors we introduced a circuit containing parts of the zeaxanthin synthesis pathway. The La Cañada subset of our team focused on tocopherol a metabolite that has similar properties to zeaxanthin in cyanobacteria. Tocopherol acts as an antioxidant protects the cell from lipid peroxidation and enhances photosynthesis. A circuit was made with the gene p-hydroxyphenylpyruvate dioxygenase to catalyze the formation of homogenistic acid the rate limiting step of tocopherol synthesis. In an attempt to further increase the production of fatty acids both zeaxanthin and tocopherol circuits are dynamically regulated utilizing a fatty acid-sensitive promoter-repressor system pLR and FadR.
\vspace{2cm} $ $
\pagebreak

\textbf{\uppercase{Reading}} \FloatBarrier \begin{table}[h] \begin{tabular}{lp{2.5cm}llll} \textbf{Location:} & United Kingdom & \multicolumn{1}{|l}{} & \textbf{Track:}   & Energy \\ \textbf{Region:}   & Europe   & \multicolumn{1}{|l}{} & \textbf{Section:} &  \\ \textbf{Year:}     & 2015   & \multicolumn{1}{|l}{} & \textbf{Awards:}  & \end{tabular} \end{table} \FloatBarrier \noindent\textbf{Innovating living photovoltaics: Renewable energy from Cyanobacteria} \vspace{.2cm}\\
Conventional photovoltaics provide a clean source of renewable energy but have the disadvantages of being expensive and containing toxic materials. This years iGEM project was to develop a cheaper non-toxic alternative to conventional fuel cells; using synthetic biology. Biological photovoltaics (BPV) are a promising candidate to provide an alternative. Our BPV uses the Cyanobacterium Synechocystis sp. PCC 6803 as the electron source. Using a purpose built fuel cell and Synechocystis which has been genetically modified to improve interactions between the bacterium and the anode surface enables the BPV to generate a greater voltage. This BPV will be considered for large scale usage in homes and communities worldwide as a cheap simple and clean alternative to conventional energy sources. 
\vspace{2cm}

\textbf{\uppercase{Uniandes\_Colombia}} \FloatBarrier \begin{table}[h] \begin{tabular}{lp{2.5cm}llll} \textbf{Location:} & Colombia & \multicolumn{1}{|l}{} & \textbf{Track:}   & Information Processing \\ \textbf{Region:}   & Latin America   & \multicolumn{1}{|l}{} & \textbf{Section:} &  \\ \textbf{Year:}     & 2015   & \multicolumn{1}{|l}{} & \textbf{Awards:}  & \end{tabular} \end{table} \FloatBarrier \noindent\textbf{Building a Bio-Electronic Clock} \vspace{.2cm}\\ 
We are currently working on a bio-digital clock as a proof-of-concept project dealing with the integration of biological and electronic circuits. We plan to modify the circadian clock Kai protein system of cyanobacteria Synechococcus elongatus by hooking it to the AHL-producing half of the Lux quorum sensing system of Vibrio fischerii. The sensing portion of the Lux system will reside in modified Shewanella oneidensis engineered to produce changes in its electrical resistance in response to changing levels of AHL using this speciess control of cytochrome production. Finally another key component of our project is the design and construction of the eletcronic hardware necessary to measure S. oneidensiss changes in electrical conductance and act as an interface between this biological circuit and any electronic circuit it is to be coupled with in this example a digital clock.
\vspace{2cm} $ $
\pagebreak

\textbf{\uppercase{Yale}} \FloatBarrier \begin{table}[h] \begin{tabular}{lp{2.5cm}llll} \textbf{Location:} & United States & \multicolumn{1}{|l}{} & \textbf{Track:}   & Foundational Advance \\ \textbf{Region:}   & North America   & \multicolumn{1}{|l}{} & \textbf{Section:} &  \\ \textbf{Year:}     & 2015   & \multicolumn{1}{|l}{} & \textbf{Awards:}  & \end{tabular} \end{table} \FloatBarrier \noindent\textbf{Developing a Framework for the Genetic Manipulation of Non-Model and Environmentally Significant Microbes} \vspace{.2cm}\\ 
We established a framework for implementing genetic manipulation techniques—specifically multiplex automated genome engineering (MAGE) and CRISPR-Cas9 systems—into non-model environmentally significant microbes using standard biological parts. The framework involves two components: (1) propagation and selection of cultures and (2) manipulation of cell genomes by MAGE and/or CRISPR. We identified design considerations for both components of the framework and experimentally validated propagation and selection considerations using cyanobacterial strain Synechococcus sp. PCC 7002 (a fast-growing cyanobacterium capable of lipid biofuel production) and Sinorhizobium tropici CIAT (a nitrogen-fixing rhizobium which forms root nodules in legume plants). We then developed a workflow for the design construction and testing of MAGE and CRISPR technologies in non-model prokaryotes. The insights we gained from validating the propagation component of our workflow will serve to improve the versatility and robustness of our framework and will inform the development of tools for genetic manipulation in other non-model organisms.
\vspace{2cm}

\textbf{\uppercase{Chalmers\_Gothenburg}} \FloatBarrier \begin{table}[h] \begin{tabular}{lp{2.5cm}llll} \textbf{Location:} & Sweden & \multicolumn{1}{|l}{} & \textbf{Track:}   & Environment \\ \textbf{Region:}   & Europe   & \multicolumn{1}{|l}{} & \textbf{Section:} &  \\ \textbf{Year:}     & 2016   & \multicolumn{1}{|l}{} & \textbf{Awards:}  & \end{tabular} \end{table} \FloatBarrier \noindent\textbf{Turning pollution into a solution} \vspace{.2cm}\\ 
Current methods of chemical synthesis from petroleum have led to great environmental disruption and continue to be a strong contributor to the emission of carbon dioxide. To overcome this problem biosynthesis is the most viable alternative. The main drawback of biosynthesis is the high price of raw material comprising over 60\% of the production cost. Our solution for this complex problem is to create a self-sustaining co-culture of microorganisms that produces its own raw material using light and carbon dioxide. Cyanobacteria provide the carbon source for the production organism which in exchange produces an essential amino acid for the cyanobacteria while creating the desired product. By making several species compatible with this synthetic symbiosis the platform will allow efficient conversion of atmospheric carbon dioxide into products in an environmentally friendly and sustainable way.
\vspace{2cm} $ $
\pagebreak

\textbf{\uppercase{CSU\_Fort\_Collins}} \FloatBarrier \begin{table}[h] \begin{tabular}{lp{2.5cm}llll} \textbf{Location:} & United States & \multicolumn{1}{|l}{} & \textbf{Track:}   & New Application \\ \textbf{Region:}   & North America   & \multicolumn{1}{|l}{} & \textbf{Section:} &  \\ \textbf{Year:}     & 2016   & \multicolumn{1}{|l}{} & \textbf{Awards:}  & \end{tabular} \end{table} \FloatBarrier \noindent\textbf{CyanoLogic: A novel modular production system in Synechocystis 6803} \vspace{.2cm}\\ 
Lights Quorum Action! Boolean logic is used in computer processes by stipulating necessary inputs to produce a desired outcome. We designed a logic gate in Synechocystis sp. PCC 6803 to optimize product production. Utilizing the Boolean operator AND gene expression accommodates the organism’s natural metabolic regulation combined with the quorum sensing mechanism from Vibrio fischeri to create an autoinduction system. With the cost of large scale production in mind our system eliminates the need for expensive induction molecules such as IPTG. Under the control of light and a quorum of cells the T7 promoter from T7 bacteriophage drives production of a wide range of products from biofuels to pharmaceuticals. CyanoLogic coming soon to a lab near you!
\vspace{2cm}

\textbf{\uppercase{Edinburgh\_OG}} \FloatBarrier \begin{table}[h] \begin{tabular}{lp{2.5cm}llll} \textbf{Location:} & United Kingdom & \multicolumn{1}{|l}{} & \textbf{Track:}   & New Application \\ \textbf{Region:}   & Europe   & \multicolumn{1}{|l}{} & \textbf{Section:} &  \\ \textbf{Year:}     & 2016   & \multicolumn{1}{|l}{} & \textbf{Awards:}  & \end{tabular} \end{table} \FloatBarrier \noindent\textbf{ExpandED: Tools For Rapid Prototyping in Non-Model Hosts} \vspace{.2cm}\\ 
Industrial biotechnology is greatly dependent on the use of model organisms such as Escherichia coli and Saccharomyces cerevisiae. The minimal cell is a future ambition of synthetic biology however there remains a vast untapped reservoir of non-model organisms each with diverse and unique traits for exploitation. It is a lack of tools designed for native producer organisms that often limits use as effective bio-factories. Today advance genetic engineering techniques such as CRISPR/Cas editing and MoClo assembly methods allow rapid strain prototyping at unprecedented ease and cost. Our team aim to demonstrate the potential speed and power of these techniques in developing three diverse platform strains and respective parts libraries for use by research groups iGEM teams commercial organisations and citizen scientists.
\vspace{2cm} $ $
\pagebreak

\textbf{\uppercase{IngenuityLab\_Canada}} \FloatBarrier \begin{table}[h] \begin{tabular}{lp{2.5cm}llll} \textbf{Location:} & Canada & \multicolumn{1}{|l}{} & \textbf{Track:}   & New Application \\ \textbf{Region:}   & North America   & \multicolumn{1}{|l}{} & \textbf{Section:} &  \\ \textbf{Year:}     & 2016   & \multicolumn{1}{|l}{} & \textbf{Awards:}  & \end{tabular} \end{table} \FloatBarrier \noindent\textbf{DNA assisted assembly of modular nanowires} \vspace{.2cm}\\ 
Our project seeks to manufacture nanostructures. By bridging the two opposite approaches together we have devised a method to create a modular nanowires. With DNA Origami self assembling properties organize the DNA strand into patterns by using the local forces to find the lowest energy configuration which is an Bottom-Up approach. To fold the DNA strand into a well defined structure using DNA staples is an example of Top-Down approach. DNA has many advantages over traditional materials such as biological compatibility low manufacturing cost and the information regarding shape and size is carried over upon replication. Individual modules are 30-40 nm long 3D Structures with hollow cavity that acts as scaffold for the Gold nanowires. They self assemble into long nanowires and we attached photosystem II protein from the Synechocystis 6803 at one end of the wire to create a high efficiency machinery to harvest solar energy.
\vspace{2cm}

\textbf{\uppercase{Marburg}} \FloatBarrier \begin{table}[h] \begin{tabular}{lp{2.5cm}llll} \textbf{Location:} & Germany & \multicolumn{1}{|l}{} & \textbf{Track:}   & New Application \\ \textbf{Region:}   & Europe   & \multicolumn{1}{|l}{} & \textbf{Section:} &  \\ \textbf{Year:}     & 2016   & \multicolumn{1}{|l}{} & \textbf{Awards:}  & \end{tabular} \end{table} \FloatBarrier \noindent\textbf{SYNDUSTRY - fuse. produce. use.} \vspace{.2cm}\\ 
Globalization radically changed the world we live in; the way we communicate and travel has become much easier. On the downside our need for resources has dramatically increased causing ecological and social problems like land-grabbing and fracking. The emergence of Synthetic Biology is initiating another bio-based industrial revolution. It is time to take the next step towards a sustainable bio-industry. In Syndustry we follow nature’s own design principles by combining the strengths of individual microorganisms for the production of valuable biochemicals. We introduce a novel ‘plug-and-play’ production platform based on artificial endosymbiosis. This system goes beyond co-culturing microbes and overcomes current production limitations in fermentation. By employing cyanobacteria capable of photosynthetic growth we achieve a self-sustainable and versatile production platform for biochemicals from carbon dioxide. Syndustry – fuse. produce. use. is the next industrial revolution and will change the face of the world as we know it today!
\vspace{2cm} $ $
\pagebreak

\textbf{\uppercase{MSU-Michigan}} \FloatBarrier \begin{table}[h] \begin{tabular}{lp{2.5cm}llll} \textbf{Location:} & United States & \multicolumn{1}{|l}{} & \textbf{Track:}   & Manufacturing \\ \textbf{Region:}   & North America   & \multicolumn{1}{|l}{} & \textbf{Section:} &  \\ \textbf{Year:}     & 2016   & \multicolumn{1}{|l}{} & \textbf{Awards:}  & \end{tabular} \end{table} \FloatBarrier \noindent\textbf{Engineering Cyanobacteria for Improved Tolerance to a Freeze/Thaw Cycle} \vspace{.2cm}\\ 
Currently the biotechnologically relevant model strain of cyanobacteria Synechococcus elongatus PCC 7942 lacks resilience to cold temperature perturbations. If large-scale operations for photosynthetic production of industrial compounds are to be realized robustness to unpredictable weather conditions must be considered. Two complementary approaches are being proposed to increase cold adaptation and resistance to freezing in S. elongatus. Previously it was shown the expression of lipid desaturase desA increases the cold-growth tolerance of S. elongatus. We now aim to improve this range by fine-tuning the expression of a riboswitch-controlled desA. We also hypothesize that introduction of SFR2 from Arabidopsis thaliana—responsible for remodeling the outer chloroplast membrane for increased freezing tolerance—will increase cellular viability to freezing events. Through this two-pronged approach we aspire to engineer a cyanobacterial strain that ultimately could be used for the production of industrially-relevant products in unreliable environment conditions.
\vspace{2cm}

\textbf{\uppercase{ULV-LC-CV}} \FloatBarrier \begin{table}[h] \begin{tabular}{lp{2.5cm}llll} \textbf{Location:} & United States & \multicolumn{1}{|l}{} & \textbf{Track:}   & Energy \\ \textbf{Region:}   & North America   & \multicolumn{1}{|l}{} & \textbf{Section:} &  \\ \textbf{Year:}     & 2016   & \multicolumn{1}{|l}{} & \textbf{Awards:}  & \end{tabular} \end{table} \FloatBarrier \noindent\textbf{In vivo production of fatty acid methyl esters in cyanobacteria utilizing the insect methyltransferase DmJHAMT } \vspace{.2cm}\\ 
Biodiesel is mainly composed of fatty acid methyl esters (FAMEs) and is a renewable energy source. Currently FAMEs are synthesized through transesterification of free fatty acids (FFAs) using a methyl donor such as methanol along with an alkaline catalyst to speed up the reaction. Both the extraction of FFAs and the chemicals used in this process are expensive. We intend to reduce the production cost of biodiesel by producing FAMEs in vivo using an insect methyltransferase called Drosophila melanogaster Juvenile hormone acid O-methyltransferase (DmJHAMT) within Synechococcus elongatus PCC 7942. DmJHAMT transfers the methyl groups from endogenous S-adenosyl-L-methionine (SAM) to FFAs therefore synthesizing FAMEs.  To minimize extraction costs we aim to induce self-lysis in Synechococcus at maximum optical densities using quorum sensing.  By regulating the production of autoinducers with promoters of various strengths we plan to tune the optical density at which gene expression is activated.
\vspace{2cm} $ $
\pagebreak

\textbf{\uppercase{UNIK\_Copenhagen}} \FloatBarrier \begin{table}[h] \begin{tabular}{lp{2.5cm}llll} \textbf{Location:} & Denmark & \multicolumn{1}{|l}{} & \textbf{Track:}   & New Application \\ \textbf{Region:}   & Europe   & \multicolumn{1}{|l}{} & \textbf{Section:} &  \\ \textbf{Year:}     & 2016   & \multicolumn{1}{|l}{} & \textbf{Awards:}  & \end{tabular} \end{table} \FloatBarrier \noindent\textbf{CosmoCrops: A modular platform for sustainable bioproduction in space} \vspace{.2cm}\\ 
Space missions face the problems that transporting mass is expensive and the needs of long expeditions are unknown in advance.  It would be revolutionary to have the capacity to manufacture various resources needed without prior knowledge of exact mission requirements.  We have designed a modular co-culture system to accomplish this: containing the cyanobacterium Synechococcus elongatus to use CO2 and light which are plentiful on Mars to produce sucrose.  This is used as a common feedstock by Bacillus subtilis to generate essential end-products.  We used polylactic acid as a proof-of-concept since 3D printers can use it for tools and machine parts.  To examine the co-culture’s practicality in extraterrestrial environments the cutting-edge Jens Martin Mars Chamber was used to test stresses including UV exposure and pressure extremes. We propose that Bacillus’s sporulation ability will enable missions to maintain libraries of strains for constructing a versatile array of materials for future space exploration. 
\vspace{2cm}

\textbf{\uppercase{WashU\_StLouis}} \FloatBarrier \begin{table}[h] \begin{tabular}{lp{2.5cm}llll} \textbf{Location:} & United States & \multicolumn{1}{|l}{} & \textbf{Track:}   & Environment \\ \textbf{Region:}   & North America   & \multicolumn{1}{|l}{} & \textbf{Section:} &  \\ \textbf{Year:}     & 2016   & \multicolumn{1}{|l}{} & \textbf{Awards:}  & \end{tabular} \end{table} \FloatBarrier \noindent\textbf{Super Cells: Overproducing ATP and Electron Donors in E. coli} \vspace{.2cm}\\
The Nitrogen Project of which this iGEM team is a part seeks to drastically reduce the quantity of nitrogen-based fertilizers used in agriculture. Soluble nitrates can “runoff” into water systems with environmental consequences such as algal blooms and human illnesses like methemoglobinemia. If nitrogenase the enzyme in soil bacteria that “fixes” nitrogen gas into usable nitrates can be inserted into plants it would eliminate the need for artificial fertilization. Before this can be done nitrogenase must first be expressed non-diazotrophic bacteria like E. coli. For proper expression however E. coli must have an excess of intracellular ATP and reduced electron donors. We worked to overexpress glycolytic kinases to increase ATP production and overexpress native and foreign electron donors to produce more reduced electron donors. Besides nitrogenase however the intracellular environment of our “super cells” may help produce other recombinant proteins.
\vspace{2cm} $ $
\pagebreak

\textbf{\uppercase{Amsterdam}} \FloatBarrier \begin{table}[h] \begin{tabular}{lp{2.5cm}llll} \textbf{Location:} & Netherlands & \multicolumn{1}{|l}{} & \textbf{Track:}   & Manufacturing \\ \textbf{Region:}   & Europe   & \multicolumn{1}{|l}{} & \textbf{Section:} &  \\ \textbf{Year:}     & 2017   & \multicolumn{1}{|l}{} & \textbf{Awards:}  & \end{tabular} \end{table} \FloatBarrier \noindent\textbf{Photosynthetic magic: Producing fumarate out of thin air using cyanobacterial cell factories} \vspace{.2cm}\\ 
Irrespective of where you come from we all share the global responsibility of ensuring that our societies are sustainable. We have been depleting the world’s resources and filling the atmosphere with abnormal levels of CO2 for too long. Our team has decided to take on this challenge by creating photosynthetic cell factories to directly use the pollutant CO2 as a resource for synthesizing fumarate - a versatile chemical traditionally manufactured from petroleum. We aim to stably produce sense and export fumarate under conditions mimicking industrial settings. Stable production is achieved by activating and evolving different metabolic modules in response to natural day/night cycles. To detect fumarate we developed a biosensor suitable for high-throughput screening. Finally to allow optimal fumarate export we investigate its transport mechanism. These efforts attracted attention from beyond academia as our cell factories may help in taking CO2 out of thin air and into a bio-based economy.
\vspace{2cm}

\textbf{\uppercase{NYMU-Taipei}} \FloatBarrier \begin{table}[h] \begin{tabular}{lp{2.5cm}llll} \textbf{Location:} & Taiwan & \multicolumn{1}{|l}{} & \textbf{Track:}   & Energy \\ \textbf{Region:}   & Asia   & \multicolumn{1}{|l}{} & \textbf{Section:} &  \\ \textbf{Year:}     & 2017   & \multicolumn{1}{|l}{} & \textbf{Awards:}  & \end{tabular} \end{table} \FloatBarrier \noindent\textbf{ Smart AlgaEnergy} \vspace{.2cm}\\
Facing the threatening energy crisis scientists are craving for alternative energy sources. Taking both clean energy productivity and other factors under consideration we have decided to target our project on increasing the oil accumulation in microalgae by multiple approaches. On the one hand we have determined to make microalgae undergo nitrogen starvation to increase its oil accumulation by creating a co-culturing system of microalgae and NrtA-transformed Escherichia coli that can deprive microalgae of nitrogen source. On the other hand we have changed the color of microalgae by transforming pigmentation functions from other species into microalgae cells to enhance its efficiency of photosynthesis. By combining these two approaches we can develop a new intelligent system which can enhance bio-energy production and contribute to the needs of renewable clean energy.
\vspace{2cm} $ $
\pagebreak

\textbf{\uppercase{UCSC}} \FloatBarrier \begin{table}[h] \begin{tabular}{lp{2.5cm}llll} \textbf{Location:} & United States & \multicolumn{1}{|l}{} & \textbf{Track:}   & Manufacturing \\ \textbf{Region:}   & North America   & \multicolumn{1}{|l}{} & \textbf{Section:} &  \\ \textbf{Year:}     & 2017   & \multicolumn{1}{|l}{} & \textbf{Awards:}  & \end{tabular} \end{table} \FloatBarrier \noindent\textbf{Bugs Without Borders} \vspace{.2cm}\\ 
Much of the world struggles with inadequate access to essential medicines and nutrition due to high pharmaceutical prices and unreliable distribution. Our solution is to decentralize production of these resources by engineering the robust cyanobacterium Arthrospira platensis to produce essential medicines and supplements self-sustainably photosynthetically and on-site at healthcare facilities. However due to insufficient research into the genetics of A. platensis we have undertaken two separate engineering endeavors in the metabolically similar Synechococcus elongatus PCC 7942 to produce acetaminophen and human-usable vitamin B12. The genes for acetaminophen production 4ABH and nhoA and those for B12 production ssuE and bluB have been transformed into PCC 7942 with the aim of method validation and subsequent migration to A. platensis once a genetic system has been established.  In addition we are performing a whole genome sequencing of A. platensis UTEX 2340 to further this research and its potential medical applications.
\vspace{2cm}

\textbf{\uppercase{UC\_San\_Diego}} \FloatBarrier \begin{table}[h] \begin{tabular}{lp{2.5cm}llll} \textbf{Location:} & United States & \multicolumn{1}{|l}{} & \textbf{Track:}   & New Application \\ \textbf{Region:}   & North America   & \multicolumn{1}{|l}{} & \textbf{Section:} &  \\ \textbf{Year:}     & 2017   & \multicolumn{1}{|l}{} & \textbf{Awards:}  & \end{tabular} \end{table} \FloatBarrier \noindent\textbf{SynEco: A Xenobiotics-derived Co-culture System of S. elongatus and E. coli for Applications in Bioproduction} \vspace{.2cm}\\ 
Currently the biofuel industry uses glucose as feedstock for E. coli. Methods to prevent contamination require expensive process sterilization or excess antibiotics dosage which become ineffective over time by promoting antibiotic resistance. In our project we use a xenobiotic approach to engineer autotrophic cyanobacteria Synechococcus elongatus PCC 7942 to produce the rare sugar D-tagatose in a five-step enzymatic pathway. After detecting tagatose via HPLC we will seek to engineer a transporter for tagatose secretion a mechanism that is currently not known. The cyanobacteria will be harnessed in conjunction with genetically engineered E. coli that can metabolize tagatose; by using a multifaceted approach we could leverage xenobiotic technology in a unique way to create a novel production platform that uses tagatose as a nutrient source and an anti-contamination agent. Because our system utilizes photosynthetic cyanobacteria to cheaply produce the rare sugar carbon source our co-culture system is both self-sustainable and cost-effective.
\vspace{2cm} $ $
\pagebreak

\textbf{\uppercase{WashU\_StLouis}} \FloatBarrier \begin{table}[h] \begin{tabular}{lp{2.5cm}llll} \textbf{Location:} & United States & \multicolumn{1}{|l}{} & \textbf{Track:}   & Environment \\ \textbf{Region:}   & North America   & \multicolumn{1}{|l}{} & \textbf{Section:} &  \\ \textbf{Year:}     & 2017   & \multicolumn{1}{|l}{} & \textbf{Awards:}  & \end{tabular} \end{table} \FloatBarrier \noindent\textbf{Operation: Ultraviolet} \vspace{.2cm}\\ 
Due to numerous climate effects photosynthetic organisms are being damaged due to increased levels of harmful UV-B radiation. Luckily many organisms exist that have a natural resistance to this kind of radiation. Using genes from the tardigrade species Ramazzottius Varieornatus the bacteria species Deinococcus Radiodurans and a strain of cyanobacteria we hope to induce resistance to UV-B radiation damage. In order to demonstrate proof of these genes’ utility our first step was to transform the genes into E. Coli. From there we planned to test these transformed cells in our Environmental Simulation System by irradiating the cells and comparing their growth to that of a control. So far we have found that E. Coli cells with Dsup are substantially more resistant to UV-B radiation and we have also recorded preliminary data on uvsE as well. Currently we are transforming the four genes into cyanobacteria and testing them under UV-B light.
\vspace{2cm}

\textbf{\uppercase{Worldshaper-Nanjing}} \FloatBarrier \begin{table}[h] \begin{tabular}{lp{2.5cm}llll} \textbf{Location:} & China & \multicolumn{1}{|l}{} & \textbf{Track:}   & High School \\ \textbf{Region:}   & Asia   & \multicolumn{1}{|l}{} & \textbf{Section:} &  \\ \textbf{Year:}     & 2017   & \multicolumn{1}{|l}{} & \textbf{Awards:}  & \end{tabular} \end{table} \FloatBarrier \noindent\textbf{Self-sinking Algae for CO2 Sequestration} \vspace{.2cm}\\ 
Increasing level of carbon dioxide in earths atmosphere is the main reason of global warming. Thus it has become important to slow down the accumulation rate or reduce the amount of carbon dioxide in atmosphere. Here we hope to develop an efficient biological carbon dioxide sequestration system using Synchronous sp. PCC 7002. The capability of CO2 capturing and storage of the algae was improved by inserting foreign genes encoding rubisco and starch synthase. The modified strain was also designed to have an expression of a metal binding protein on its pilus to increase weight and deactivate pilus slowly which would finally cause the alga to sink to the sea bottom permanently so as to cut off the carbon from being reused.
\vspace{2cm} $ $
\pagebreak

\textbf{\uppercase{Duesseldorf}} \FloatBarrier \begin{table}[h] \begin{tabular}{lp{2.5cm}llll} \textbf{Location:} & Germany & \multicolumn{1}{|l}{} & \textbf{Track:}   & Foundational Advance \\ \textbf{Region:}   & Europe   & \multicolumn{1}{|l}{} & \textbf{Section:} &  \\ \textbf{Year:}     & 2018   & \multicolumn{1}{|l}{} & \textbf{Awards:}  & \end{tabular} \end{table} \FloatBarrier \noindent\textbf{Trinity - towards an engineered co-culture toolbox} \vspace{.2cm}\\ 
Co-cultures are found in all conceivable entities such as the human gut cheese or plants but good tools to study those communities are currently not given. Indeed we created a modularly built toolbox using not only three different dependencies but also three different organisms: With Escherichia coli Saccharomyces cerevisiae and Synechococcus elongatus our team engineered a system based on nutrient exchange. Here phosphate is provided through oxidation of phosphite nitrogen source produced by melamine breakdown whilst carbon source is provided by Synechococcus elongatus. Two additional independent approaches are designed too. The first includes regulation via cross-feeding by amino acid auxotrophies and production: lysine by Escherichia coli and leucine by Saccharomyces cerevisiae. The other utilizes regulated self-lysis via quorum sensing molecules to control cell density by a phage lysis gene. This engineered toolbox opens a wide range of possibilities to create microbial communities for different purposes such as synthetic probiotics.
\vspace{2cm}

\textbf{\uppercase{NU\_Kazakhstan}} \FloatBarrier \begin{table}[h] \begin{tabular}{lp{2.5cm}llll} \textbf{Location:} & Kazakhstan & \multicolumn{1}{|l}{} & \textbf{Track:}   & Environment \\ \textbf{Region:}   & Asia   & \multicolumn{1}{|l}{} & \textbf{Section:} &  \\ \textbf{Year:}     & 2018   & \multicolumn{1}{|l}{} & \textbf{Awards:}  & \end{tabular} \end{table} \FloatBarrier \noindent\textbf{From a Dangerous Waste to Functional Nanomaterials: Bioremediation of Sour Crude Oil Waste using Cyanobacteria} \vspace{.2cm}\\ 
Accumulation of a hydrogen sulfide as a consequence of sulfur-containing “sour” oil refinement can be dangerous. H2S damages the drilling equipment and causes corrosion of transporting pipelines. We use Cyanobacteria as a chassis since the organism is autotrophic. We designed a Synechococcus elongatus PCC 7942 that expresses Sulfide Quinone Reductase (SQR) that catalyzes sulfide-dependent plastoquinone reduction in anaerobic conditions while photosystem II stays inhibited due to sulfide being present. SQR converts Sulfide to elemental Sulfur which is stored in the bacteria and accumulates in the Biomass. The electron flow in this modified Photosynthetic Electron Transport Chain goes to a transgenic Hydrogenase making use of the existing anoxygenic conditions due to sulfide presence. The Biomass is finally converted to functional materials used for Proton Exchange Membrane (PEM) fuel cells in accordance with a newly developed method in our laboratory.
\vspace{2cm} $ $
\pagebreak

\textbf{\uppercase{SCAU-China}} \FloatBarrier \begin{table}[h] \begin{tabular}{lp{2.5cm}llll} \textbf{Location:} & China & \multicolumn{1}{|l}{} & \textbf{Track:}   & New Application \\ \textbf{Region:}   & Asia   & \multicolumn{1}{|l}{} & \textbf{Section:} &  \\ \textbf{Year:}     & 2018   & \multicolumn{1}{|l}{} & \textbf{Awards:}  & \end{tabular} \end{table} \FloatBarrier \noindent\textbf{Desertification combating strategy: bacterial cellulose biosynthesis in desert surviving cyanobacteria} \vspace{.2cm}\\ 
Desertification is becoming a serious global problem. Great efforts have been put into the desertification control by introducing various methods. Here we take advantage of using genetic engineering and synthetic biology as powerful tools to propose a new strategy for the densification control. We use Acetobacter xylinus which is a model bacterium for producing cellulose. Its cellulose can be used for water conserving both soil and moisture. On the other hand Microcolus vaginatus is a dry land living cyanobacteria which is an ideal bioreactor for producing bacterial cellulose. We cloned seven key genes that are critically required for bacterial cellulose synthesis from Acetobacter xylinus and expressed them in cyanobacteria. Additionally we employed computer modeling and prediction to optimized the production of cellulose. Finally we successfully achieved the cellulose production from the transgenic cyanobacteria and its cultivation on sands. Together we have developed a new and low-cost method for desertification control. 
\vspace{2cm}

\textbf{\uppercase{Stony\_Brook}} \FloatBarrier \begin{table}[h] \begin{tabular}{lp{2.5cm}llll} \textbf{Location:} & United States & \multicolumn{1}{|l}{} & \textbf{Track:}   & Energy \\ \textbf{Region:}   & North America   & \multicolumn{1}{|l}{} & \textbf{Section:} &  \\ \textbf{Year:}     & 2018   & \multicolumn{1}{|l}{} & \textbf{Awards:}  & \end{tabular} \end{table} \FloatBarrier \noindent\textbf{The Sucrose Factory} \vspace{.2cm}\\ 
In 2017 humans released ~32.5 gigatons of CO2 into the atmosphere. Even if anthropogenic carbon emissions ended today the CO2 in our atmosphere would persist for thousands of years causing ocean acidification and global warming. Current carbon sink technology is not economically feasible and would cost trillions of dollars at modest estimates. We believe the solution lies in cyanobacteria - photosynthetic prokaryotes - as they were the first organisms to sink carbon dioxide billions of years ago and are some of the most efficient autotrophs. Our approach is to induce sucrose secretion for the industrial production of biofuels and bioplastics while simultaneously sinking CO2. Additionally to address the lack of promoters available for cyanobacteria synthetic biology research our team developed a variety of constitutive light-inducible and nutrient-repressible promoter BioBricks for our strain of Synechococcus elongatus. We hope these promoters will be used to produce other high value carbon sinking products.
\vspace{2cm} $ $
\pagebreak

\textbf{\uppercase{UChicago}} \FloatBarrier \begin{table}[h] \begin{tabular}{lp{2.5cm}llll} \textbf{Location:} & United States & \multicolumn{1}{|l}{} & \textbf{Track:}   & New Application \\ \textbf{Region:}   & North America   & \multicolumn{1}{|l}{} & \textbf{Section:} &  \\ \textbf{Year:}     & 2018   & \multicolumn{1}{|l}{} & \textbf{Awards:}  & \end{tabular} \end{table} \FloatBarrier \noindent\textbf{An iGEM-Optimized CEN Plasmid for E. coli and Pichia pastoris} \vspace{.2cm}\\ 
Komatgella pastoris otherwise known as Pichia pastoris serves as an important industrial chasis organism for its ease of cultivation while also making post transcriptional modifications to eukaryotic proteins. Expensive and complex techniques such as in vivo recombination however remain a major bottleneck to developing transgenic P. pastoris lines. Centromeric plasmids developed for Saccharomyces cerevisiae overcome this bottleneck by providing the flexibility of plasmids with the stability of endogenous chromosomes. Here we adapt the pSB1C3 iGEM backbone with a P. pastoris selection marker and various portions of the P. pastoris centromeric sequences to develop centromeric plasmids. We demonstrate by sectoring assay that these plasmids provide chromosome-like stability while maintaining the ease of use of an iGEM plasmid. This plasmid has major implications in the  manufacturing of biologics.
\vspace{2cm}

\textbf{\uppercase{Duesseldorf}} \FloatBarrier \begin{table}[h] \begin{tabular}{lp{2.5cm}llll} \textbf{Location:} & Germany & \multicolumn{1}{|l}{} & \textbf{Track:}   & Food & Nutrition \\ \textbf{Region:}   & Europe   & \multicolumn{1}{|l}{} & \textbf{Section:} &  \\ \textbf{Year:}     & 2019   & \multicolumn{1}{|l}{} & \textbf{Awards:}  & \end{tabular} \end{table} \FloatBarrier \noindent\textbf{SynMylk - an eco-friendly synthetic cow’s milk to save the environment } \vspace{.2cm}\\ 
Our project is the production of the natural components of cow’s milk using methods from synthetic biology to modify microorganisms. This solution can provide the world with milk without risking the environmental damage caused by massive animal farms while providing an authentic alternative. This lactose-free milk will be available to a larger number of people around the world.The first step to creating our SynMylk is the production of the components of cow’s milk that the chemical industry cannot provide without using animal products. These components are the milk’s proteins and lipids. We modified Bacillus subtilis Pichia pastoris and the photosynthetic cyanobacterium Synechocystis sp. PCC 6803 to produce the milk proteins heterologously. The synthesis of lipids is enhanced by overexpressing enzymes that are bottlenecks in Synechocystis’ natural fatty acid production. Heterologous enzymes are also expressed to specifically obtain certain lengths of lipids which are not naturally produced.
\vspace{2cm} $ $
\pagebreak

\textbf{\uppercase{HK\_SSC}} \FloatBarrier \begin{table}[h] \begin{tabular}{lp{2.5cm}llll} \textbf{Location:} & Hong Kong & \multicolumn{1}{|l}{} & \textbf{Track:}   & High School \\ \textbf{Region:}   & Asia   & \multicolumn{1}{|l}{} & \textbf{Section:} &  \\ \textbf{Year:}     & 2019   & \multicolumn{1}{|l}{} & \textbf{Awards:}  & \end{tabular} \end{table} \FloatBarrier \noindent\textbf{Expression of dCas9-sgRNA Complex in Microcystis Aeruginosa Resulting in the Repression of its Toxin-producing Gene} \vspace{.2cm}\\ 
Microcystis aeruginosa is one of the most common cyanobacteria responsible for harmful algal blooms. This cyanobacterium produces microcystin a hepatotoxin that damages the liver. However direct lysis of Microcystis aeruginosa may not best for the environment as it holds ecological values of heavy metal sorption and oxygen synthesis. We hope to silence the microcystin biosynthesis cluster(mcy) using a catalytically dead Cas9 (dCas9) enzyme lacking endonuclease activity. When the dCas9 enzyme is co-expressed with a guide RNA(sgRNA) the dCas9-sgRNA complex specifically binds to the McyB gene and blocks transcript elongation leading to the repression of the McyB gene without altering the chromosome of the Microcystis. Here we provide the design of a dCas9-sgRNA expression gene in a shuttle vector that can replicate in both E.coli and cyanobacteria. We will also be conducting downstream analysis to see how our dCas9-sgRNA expression plasmid affects the microcystin-production rate and oxygen synthesis rate of Microcystis.
\vspace{2cm}

\textbf{\uppercase{IIT\_Chicago}} \FloatBarrier \begin{table}[h] \begin{tabular}{lp{2.5cm}llll} \textbf{Location:} & United States & \multicolumn{1}{|l}{} & \textbf{Track:}   & Environment \\ \textbf{Region:}   & North America   & \multicolumn{1}{|l}{} & \textbf{Section:} &  \\ \textbf{Year:}     & 2019   & \multicolumn{1}{|l}{} & \textbf{Awards:}  & \end{tabular} \end{table} \FloatBarrier \noindent\textbf{Green Ocean} \vspace{.2cm}\\ 
Green Ocean’s aim is to genetically modify marine cyanobacteria that will enable it to degrade polyethylene terephthalate (PET) most common form of plastic in the oceans. The engineered cyanobacteria harbor PETase an enzyme that breaks down PET. Our approach is novel because instead of using the traditional e. coli which may not survive in the ocean environment cyanobacteria are photosynthetic bacteria that thrive in the ocean. We have modified the prototypical Ideonella sakaiensis  PETase gene to be compatible with expression and secretion in cyanobacteria. This engineering was accomplished in a dual-host plasmid shuttle vector in E coli and then transferred to a model cyanobacterium Synechococcus elongatus by conjugation.We also developed a PET degradation assay system consisting of fluorescent PET nanoparticles. The degradation of the PET nanoparticles was measured by a variety of imaging and functional assays. We desire to make a change in the world starting with a Green Ocean. 
\vspace{2cm} $ $
\pagebreak

\textbf{\uppercase{ITESO\_Guadalajara}} \FloatBarrier \begin{table}[h] \begin{tabular}{lp{2.5cm}llll} \textbf{Location:} & Mexico & \multicolumn{1}{|l}{} & \textbf{Track:}   & Environment \\ \textbf{Region:}   & Latin America   & \multicolumn{1}{|l}{} & \textbf{Section:} &  \\ \textbf{Year:}     & 2019   & \multicolumn{1}{|l}{} & \textbf{Awards:}  & \end{tabular} \end{table} \FloatBarrier \noindent\textbf{RubisCO} \vspace{.2cm}\\
In RubisCO we are thinking of new ways in which we can manage the waste we put in the environment through the gas and wastewater streams that come from the city and the industry by harnessing the capability of cyanobacteria to grow in brackish water and to fix carbon dioxide through its metabolism. But this process has become slow and prone to errors losing part of its output through photorespiration. From this understanding we are focusing on enhancing the carbon fixing mechanisms of Synechococcus sp. and conducting the surplus of carbon flow to the synthesis of high added-value chemical intermediates such as free fatty acids to increase the economic feasibility of the implementation of  Carbon Capture and Utilization technologies which are urgently needed to fight back Climate Change. Systems Biology Bioprocess’ Simulation and integral stakeholder management have been performed to assess the feasibility and impact of the proposal here presented.
\vspace{2cm}

\textbf{\uppercase{KU\_LEUVEN}} \FloatBarrier \begin{table}[h] \begin{tabular}{lp{2.5cm}llll} \textbf{Location:} & Belgium & \multicolumn{1}{|l}{} & \textbf{Track:}   & Manufacturing \\ \textbf{Region:}   & Europe   & \multicolumn{1}{|l}{} & \textbf{Section:} &  \\ \textbf{Year:}     & 2019   & \multicolumn{1}{|l}{} & \textbf{Awards:}  & \end{tabular} \end{table} \FloatBarrier \noindent\textbf{OCYANO - The development of two low-input photosynthetic systems for sustainable protein production} \vspace{.2cm}\\ 
Traditional biosynthesis platforms such as E. coli and yeast require external energy supplies commonly in the form of sugars or starch. Besides the economic cost associated with these energy sources such systems are often not considered durable. Indeed the production processes of sugars and starch are energy inefficient and farmland intensive. To circumvent these issues photosynthetic systems like cyanobacteria and algae have been gaining increasing interest for biosynthetic purposes as they require only light and CO2. With our project OCYANO we present two new cyanobacterial technologies for protein production. The first design comprises the production and secretion of proteins in an ultra-fast growing cyanobacterium. The second system relies on a cyanophage for the conversion of its host’s biomass to the protein of interest. Along with wet-lab exploration of these platforms the economic and ecological relevances of both systems were investigated and compared to state of the art biosynthesis platforms.)
\vspace{2cm} $ $
\pagebreak

\textbf{\uppercase{Lethbridge}} \FloatBarrier \begin{table}[h] \begin{tabular}{lp{2.5cm}llll} \textbf{Location:} & Canada & \multicolumn{1}{|l}{} & \textbf{Track:}   & Manufacturing \\ \textbf{Region:}   & North America   & \multicolumn{1}{|l}{} & \textbf{Section:} &  \\ \textbf{Year:}     & 2019   & \multicolumn{1}{|l}{} & \textbf{Awards:}  & \end{tabular} \end{table} \FloatBarrier \noindent\textbf{Algulin: a low-cost oral insulin produced and administered in microalgae.} \vspace{.2cm}\\ 
Diabetes a disease caused by abnormal insulin regulation and production affects approximately 8.8\% of the population. Currently subcutaneous injection of recombinant insulin is used to self-regulate abnormal blood glucose levels a treatment that is painful and often prohibitively expensive for patients. Oral insulin alternatives are not yet a cost-effective alternative because the unprotected insulin is rapidly degraded by acidic stomach conditions and so there remains an unmet demand for low-cost methods of manufacturing oral insulin and/or novel methods for delivering insulin directly to the intestines. We are developing an edible recombinant microalgae strain called “Algulin” that produces either an ultrastable oral insulin analog or proinsulin peptides. Algulin reduces manufacturing costs by eliminating the need for insulin extraction and purification improves efficacy over previous oral insulins by acting as a protective capsule and shielding the insulin from degradation and eliminates uncomfortable injections for diabetic patients. 
\vspace{2cm}

\textbf{\uppercase{Marburg}} \FloatBarrier \begin{table}[h] \begin{tabular}{lp{2.5cm}llll} \textbf{Location:} & Germany & \multicolumn{1}{|l}{} & \textbf{Track:}   & Foundational Advance \\ \textbf{Region:}   & Europe   & \multicolumn{1}{|l}{} & \textbf{Section:} &  \\ \textbf{Year:}     & 2019   & \multicolumn{1}{|l}{} & \textbf{Awards:}  & \end{tabular} \end{table} \FloatBarrier \noindent\textbf{Green Revolution - Establishing the fastest growing photothrophic organism as a chassis for synthetic biology} \vspace{.2cm}\\ 
While most iGEM teams were working with conventional chassis like E. coli and S. cerevisiae phototrophic organisms were always underrepresented. To make it more feasible for other teams to work with phototrophic organisms a fast growing and easy to handle chassis is necessary. For this purpose we establish Synechococcus elongatus UTEX 2973 with a reported doubling time of 90min - as a viable chassis by developing strains tailored to various applications. Therefore we restore its natural competence establish the CRISPR/Cpf1 system for multiplexed genome engineering and enable the utilization of plasmids as a tool for rapid design testing. Furthermore we expand last years’ Golden Gate based MoClo toolbox and accelerate the complete cloning workflow by automating plating colony picking and plasmid purification on the Opentrons OT-2. By providing our fast phototrophic chassis to the community we would like to pave the way for other phototrophic organisms in synthetic biology.
\vspace{2cm} $ $
\pagebreak

\textbf{\uppercase{Mingdao}} \FloatBarrier \begin{table}[h] \begin{tabular}{lp{2.5cm}llll} \textbf{Location:} & Taiwan & \multicolumn{1}{|l}{} & \textbf{Track:}   & High School \\ \textbf{Region:}   & Asia   & \multicolumn{1}{|l}{} & \textbf{Section:} &  \\ \textbf{Year:}     & 2019   & \multicolumn{1}{|l}{} & \textbf{Awards:}  & \end{tabular} \end{table} \FloatBarrier \noindent\textbf{Indoor Air Freshener 2.0} \vspace{.2cm}\\ 
Indoor air pollution could be worse than outdoor air. That’s why people buy air purifiers at home. Yet CO2 and VOCs cannot be eliminated by any current machine. Algae purification system is increasingly getting attention but with limited efficiency. This year we improve the system significantly by combining a photobioreactor device and algae culture media supplemented with natural enzymes. We produce carbonic anhydrase (CA) to enhance CO2 dissolving rate as well as CYP2E1 to break down chloroform and benzene. The resulting molecules can easily be taken up by algae. Our device sets up with a nano bubble generator high power LED light and CO2/O2 sensors to optimize photosynthesis and analyze air quality and as small as a portable 1L water bottle.  In addition we used mathematical modeling to simulate the application in the real world. We believe it will be the most common air purifier in our life.     
\vspace{2cm}

\textbf{\uppercase{NU\_Kazakhstan}} \FloatBarrier \begin{table}[h] \begin{tabular}{lp{2.5cm}llll} \textbf{Location:} & Kazakhstan & \multicolumn{1}{|l}{} & \textbf{Track:}   & Energy \\ \textbf{Region:}   & Asia   & \multicolumn{1}{|l}{} & \textbf{Section:} &  \\ \textbf{Year:}     & 2019   & \multicolumn{1}{|l}{} & \textbf{Awards:}  & \end{tabular} \end{table} \FloatBarrier \noindent\textbf{Circular BioEconomy: How Toxic Waste is converted into Nano-electrocatalysts and Fuel} \vspace{.2cm}\\ A 
Our project is focused on production of Hydrogen gas using transformed cyanobacteria Synechococcus Elongatus PCC 7942. We introduce 3 genes: HydA HydG and HydEF. HydA is [Fe-Fe] Hydrogenase and other two proteins are maturation proteins. To improve production of hydrogen our team came up with several modification. First is to indroduce bacterial Rhodopsin that will pump protons to the site where peripheral HydA resides. Furthermore favorably fluorescent Carbon Quantum Dots can be added to redirect energy of light to rhodopsin thus increasing its pumping rate. Previously introduced SQR also can be used in this case to substitute for inactivated bu sulfide wastewater PSII providing protons and electrons from sulfide. Ultimately all biomass will be converted into graphitic catalytic material that can be use as substitute for platinum catalyst in PEM.
\vspace{2cm} $ $
\pagebreak

\textbf{\uppercase{PuiChing\_Macau}} \FloatBarrier \begin{table}[h] \begin{tabular}{lp{2.5cm}llll} \textbf{Location:} & Macao & \multicolumn{1}{|l}{} & \textbf{Track:}   & High School \\ \textbf{Region:}   & Asia   & \multicolumn{1}{|l}{} & \textbf{Section:} &  \\ \textbf{Year:}     & 2019   & \multicolumn{1}{|l}{} & \textbf{Awards:}  & \end{tabular} \end{table} \FloatBarrier \noindent\textbf{To Develop A Sustainable System For Endocrine Disrupting Chemicals Degradation} \vspace{.2cm}\\ 
Our project aims at solving the Endocrine Disrupting Chemicals (EDCs) water pollution problem. EDC is a collection of chemicals that have long-lasting negative impact on human. EDC exposure is linked to diseases such as cancers and neurodegenerative disorders. Previous studies suggested that laccase can degrade various EDCs. In this project we used engineered E.coli BL21 (DE3) to produce our selected Laccases. We cloned a collection of Laccases into E. coli which include a stress-tolerant laccase. To develop a sustainable EDC degradation system we also added a secretion signal peptide NSP4 to the laccases expressed in E. coli. Moreover we also built a green laccase production system. We transformed the laccases with a PilA secretion signal peptide into cyanobacteria (Synechococcus sp). In addition we also designed a water filter that fits our engineered bacteria. All together we believe that our project can help to find a solution for EDC water pollution.
\vspace{2cm}

\textbf{\uppercase{SZTA\_Szeged\_HU}} \FloatBarrier \begin{table}[h] \begin{tabular}{lp{2.5cm}llll} \textbf{Location:} & Hungary & \multicolumn{1}{|l}{} & \textbf{Track:}   & High School \\ \textbf{Region:}   & Europe   & \multicolumn{1}{|l}{} & \textbf{Section:} &  \\ \textbf{Year:}     & 2019   & \multicolumn{1}{|l}{} & \textbf{Awards:}  & \end{tabular} \end{table} \FloatBarrier \noindent\textbf{Detecting microcystin production of the harmful algae Microcystis aeruginosa} \vspace{.2cm}\\ 
Microcystis is a genus of cyanobacteria frequently causing harmful algal blooms and water toxicity. Our purpose is to detect the presence of microcystin a hepatotoxin produced by Microcystis aeruginosa under certain conditions. Microcystin is synthesized nonribosomally via microcystin synthetase encoded by the mcy genes. We have constructed plasmids where after the promoter region mcy genes are replaced with GFP genes. We would like to transform the plasmids into M. aeruginosa and Escherichia coli using shuttle plasmids. Upon addition of the transformed bacteria to wild-type M. aeruginosa cultures we expect that the inserted GFP genes will be transcribed due to cell-to-cell communication. By taking samples from the growing cultures we can determine the algae concentration which microcystin starts to be produced at. For further studies since its sequence is unknown we are going to sequence the promoter of mcy genes of Microcystis flos-aquae another species abundant in Hungarian lakes.
\vspace{2cm} $ $
\pagebreak

\textbf{\uppercase{TU\_Dresden}} \FloatBarrier \begin{table}[h] \begin{tabular}{lp{2.5cm}llll} \textbf{Location:} & Germany & \multicolumn{1}{|l}{} & \textbf{Track:}   & Diagnostics \\ \textbf{Region:}   & Europe   & \multicolumn{1}{|l}{} & \textbf{Section:} &  \\ \textbf{Year:}     & 2019   & \multicolumn{1}{|l}{} & \textbf{Awards:}  & \end{tabular} \end{table} \FloatBarrier \noindent\textbf{DipGene – Designing a Gene-Sensitive Paper Strip} \vspace{.2cm}\\ 
The identification of specific DNA sequences is needed in many contexts. Its applications range from testing for genetic diseases or viruses that integrate into the human genome to checking for the presence of antibiotic resistances in pathogens. Current state of the art methods are expensive slow and require advanced technologies which make genetic testing only accessible to researchers and not to most of humanity. We aim to provide a tool for detecting any nucleic acid sequence of interest from microbial samples and human cells. By combining a novel DNA extraction method with a newly designed fusion protein it will be possible to obtain a visual color readout within minutes which will indicate the presence or absence of the sequence of interest. Our method is designed to be utilized in the field meaning it will be cheap fast and easy-to-use and will not require any advanced technologies or electricity. 
\vspace{2cm}

\textbf{\uppercase{Amsterdam}} \FloatBarrier \begin{table}[h] \begin{tabular}{lp{2.5cm}llll} \textbf{Location:} & Netherlands & \multicolumn{1}{|l}{} & \textbf{Track:}   & Information Processing \\ \textbf{Region:}   & Europe   & \multicolumn{1}{|l}{} & \textbf{Section:} &  \\ \textbf{Year:}     & 2020   & \multicolumn{1}{|l}{} & \textbf{Awards:}  & \end{tabular} \end{table} \FloatBarrier \noindent\textbf{Forbidden FRUITS: stable microbial production strategies for non-native compounds} \vspace{.2cm}\\ 
Genetically engineered cellular systems can be used to produce industrially valuable compounds in a sustainable way. A challenge is that it is more beneficial for cells to use their resources exclusively for growth resulting in a loss of production ability. Therefore we have developed Forbidden FRUITS a pipeline that can solve this problem by calculating and optimizing engineering strategies to couple a product forming pathway to microbial growth. Multiple databases constraint-based programming and gene-protein-reaction associations are used to devise suitable strategies. These strategies are then optimized using pathfinding methods and sequence optimization. As proof-of-principle we applied Forbidden FRUITS to salicylic acid lactate and mannitol production in Synechocystis PCC6803 lactate in Synechococcus UTEX 2973 and salicylic acid in Escherichia Coli. Forbidden FRUITS is shown to be flexible and allow for the fast development of stable production strains making the full-potential of biotechnology evermore attainable.
\vspace{2cm} $ $
\pagebreak

\textbf{\uppercase{Baltimore\_BioCrew}} \FloatBarrier \begin{table}[h] \begin{tabular}{lp{2.5cm}llll} \textbf{Location:} & United States & \multicolumn{1}{|l}{} & \textbf{Track:}   & High School \\ \textbf{Region:}   & North America   & \multicolumn{1}{|l}{} & \textbf{Section:} &  \\ \textbf{Year:}     & 2020   & \multicolumn{1}{|l}{} & \textbf{Awards:}  & \end{tabular} \end{table} \FloatBarrier \noindent\textbf{Improving Iron Uptake and Processing in Synechococcus CB0101 to Bolster Marine Ecosystems} \vspace{.2cm}\\  In 1/3 of the world’s oceans the iron concentration limits phytoplankton growth. Iron is required for photosynthesis and is critical for the base of the marine food web. A better ability to capture iron could increase phytoplankton populations which would have benefits such as reducing atmospheric carbon dioxide by acting as a carbon sink. We decided to engineer Synechococcus (cyanobacteria) because it consumes high levels of CO2 has a high replication rate and has been used by many previous iGEM teams. Our project will engineer cyanobacteria to transport iron into cells and reduce it to the bioavailable Fe(II) form.  The increased iron utilization  will increase photosynthesis and growth of phytoplankton. To prevent harmful phytoplankton blooms a kill switch will also be added to the cells to prevent overgrowth of cells if iron concentration increased significantly. These modifications will stabilize the marine food chain and absorb CO2 from the atmosphere. 
\vspace{2cm}

\textbf{\uppercase{CSU\_CHINA}} \FloatBarrier \begin{table}[h] \begin{tabular}{lp{2.5cm}llll} \textbf{Location:} & China & \multicolumn{1}{|l}{} & \textbf{Track:}   & Environment \\ \textbf{Region:}   & Asia   & \multicolumn{1}{|l}{} & \textbf{Section:} &  \\ \textbf{Year:}     & 2020   & \multicolumn{1}{|l}{} & \textbf{Awards:}  & \end{tabular} \end{table} \FloatBarrier \noindent\textbf{Clean the Contamination of Cadmium ( CcC )} \vspace{.2cm}\\  Recently the cadmium-contaminated rice circulating in the market. Long-term intake of cadmium will jeopardize human bodies. To deal with the problem we utilize engineered Synechocystis as a competent cadmium absorber. Moreover with blue-ray/antitoxin suicide system the reformed alga will be appropriately contained. The cadmium can be recycled as the microorganisms will be calcined after absorption. The application of engineered alga will minimize human potential cadmium intake.
\vspace{2cm} $ $
\pagebreak

\textbf{\uppercase{DeNovocastrians}} \FloatBarrier \begin{table}[h] \begin{tabular}{lp{2.5cm}llll} \textbf{Location:} & Australia & \multicolumn{1}{|l}{} & \textbf{Track:}   & Environment \\ \textbf{Region:}   & Asia   & \multicolumn{1}{|l}{} & \textbf{Section:} &  \\ \textbf{Year:}     & 2020   & \multicolumn{1}{|l}{} & \textbf{Awards:}  & \end{tabular} \end{table} \FloatBarrier \noindent\textbf{Engineering microbes to detect and degrade pollutants} \vspace{.2cm}\\ 
Through our project we hope to eliminate benzene in polluted environments using bioremediation a process which utilisesmicrobes to degradefhazardous substances. Compared to traditional remediation practices bioremediation is cheaper and more sustainable. First we are creating a biosensor that will easily detect and measure environmental levels of benzene and catechol through a fluorescent protein expression system. Next our project will identify and isolate the benABCDE gene cluster (benzene transport and degradation genes) from specialized bacteria that naturally import benzene into their cells and break it down into energy intermediates for growth in contaminated environments. Following this we will insert these genes into a plasmid cloning vector and transform the model laboratory species Escherichia coli into a practically useful benzene degrader to clean up polluted sites on land and in water.
\vspace{2cm}

\textbf{\uppercase{XJTU-China}} \FloatBarrier \begin{table}[h] \begin{tabular}{lp{2.5cm}llll} \textbf{Location:} & China & \multicolumn{1}{|l}{} & \textbf{Track:}   & Environment \\ \textbf{Region:}   & Asia   & \multicolumn{1}{|l}{} & \textbf{Section:} &  \\ \textbf{Year:}     & 2020   & \multicolumn{1}{|l}{} & \textbf{Awards:}  & \end{tabular} \end{table} \FloatBarrier \noindent\textbf{Sand Fixers Alliance} \vspace{.2cm}\\   Upon the excessive deforestation grazing and reclamation of human beings desertification has been intensified. A natural sand-fixing system biological soil crusts was discovered to fight for desertification. But this natural sand fixation strategy always has little effect when facing the aggressive sand. Thus in our project an engineered Bacillus subtilis was constructed to effectively produce extracellular polysaccharide the key component of soil crust via introducing different combinations of key enzymes GalU and PGM. An arabinose-regulated suicide switch was also build to initiate suicide once the engineered bacteria release from the desert environment for biosafety. Furthermore a symbiotic system of engineered Bacillus subtilis and cyanobacteria was developed to form sand fixers alliance fighting for desertification. Our project is committed to educating the public about the current situation hazards and solutions of desertification and to providing a more convenient and effective strategy for desertification control.
\pagebreak
\section{Higher Plants}
\textbf{\uppercase{Cambridge-JIC}}
\FloatBarrier
\begin{table}[h]
\begin{tabular}{lp{2.5cm}llll}
\textbf{Location:} & United Kingdom & \multicolumn{1}{|l}{} & \textbf{Track:}   & Hardware \\
\textbf{Region:}   & Europe   & \multicolumn{1}{|l}{} & \textbf{Section:} &  \\
\textbf{Year:}     & 2015   & \multicolumn{1}{|l}{} & \textbf{Awards:}  &
\end{tabular}
\end{table}
\FloatBarrier
\noindent	\textbf{OpenScope - Open-source, 3D printable fluorescence microscope} \vspace{.2cm}\\
Fluorescence microscopy has revolutionised Synthetic Biology, bringing with it high costs. Cheap, adaptable, and compact; OpenScope is a novel alternative. It is a 3D-printable, low-cost digital microscope powered by Raspberry Pi© and Arduino©. The microscope supports bright-field and fluorescence modes with a resolution of four microns. OpenScope is suitable for teaching, use in developing countries and incorporation into laboratory systems. OpenScope is accompanied by a versatile, user-friendly software package. MicroMaps integrates the microscope on a remodeled, motorised translation stage. The software utilises a simple user interface similar to Google Maps©, designed by exploiting background image processing, annotation, and stitching; providing autonomous cell screening. The project was initially tested with Marchantia. Users will easily be able to create and customise programs for other organisms and screening criteria. OpenScope and MicroMaps are easily reproducible following the open source documentation.
\vspace{2cm}

\textbf{\uppercase{Georgia\_State}}
\FloatBarrier
\begin{table}[h]
\begin{tabular}{lp{2.5cm}llll}
\textbf{Location:} & United States & \multicolumn{1}{|l}{} & \textbf{Track:}   & Manufacturing \\
\textbf{Region:}   & North America   & \multicolumn{1}{|l}{} & \textbf{Section:} &  \\
\textbf{Year:}     & 2015   & \multicolumn{1}{|l}{} & \textbf{Awards:}  &
\end{tabular}
\end{table}
\FloatBarrier
\noindent	\textbf{Protein Products from Plants and Pichia: Novel Manufacturing of Analgesics and Cannabinoids} \vspace{.2cm}\\
Cannabinoids and opiates are widely used classes of pharmaceuticals; unfortunately, these drugs have strong psychoactive effects or can be addictive. Our project consists of two ideas, both revolving around utilizing bioengineered microorganisms to create non-psychoactive cannabinoids and non-addictive analgesics. To achieve this we developed two projects: (1) Manufacturing a protein expression system to produce CBDA synthase in tobacco plants using agrobacterium, (2) Engineering the pGAPα vector system to express the mambalgin in Pichia Pastoris as a continuation of the 2014 GSU iGEM project. Simultaneously, we developed a proof of concept using horseradish peroxidase. By the end of this project we hope to have produced a synthetic biological system to manufacture pharmaceutical alternatives for patients that suffer from diseases such as epilepsy, cancer, or chronic severe pain.
\vspace{2cm} $ $
\pagebreak

\textbf{\uppercase{NRP-UEA-Norwich}}
\FloatBarrier
\begin{table}[h]
\begin{tabular}{lp{2.5cm}llll}
\textbf{Location:} & United Kingdom & \multicolumn{1}{|l}{} & \textbf{Track:}   & Food & Nutrition \\
\textbf{Region:}   & Europe   & \multicolumn{1}{|l}{} & \textbf{Section:} &  \\
\textbf{Year:}     & 2015   & \multicolumn{1}{|l}{} & \textbf{Awards:}  &
\end{tabular}
\end{table}
\FloatBarrier
\noindent	\textbf{Engineering nutrition to increase colonic butryrate} \vspace{.2cm}\\
Colon cancer is the second most common cause of cancer deaths with 30,000 cases diagnosed every year in the United Kingdom. Studies suggest that resistant starches may reduce colon cancer by enabling colonic bacteria to produce short-chain fatty acids, including butyrate.
Our project took two approaches to increase colonic butyrate. The first approach was to develop a screen for enzymes that could transfer acyl/butyryl groups to alpha 1,4 carbohydrates in bacteria and plants. To support this we modelled and modified carbohydrate branching. Enzymatic modification of carbohydrates could also provide environmentally-friendly methods for the production of modified starches used in a wide range of industries. The second approach aimed to transfer the butyrate biosynthetic pathway to Escherichia coli.
Our work could be applied to the production of butyrylated starches for consumption as prebiotics or butyrate-producing probiotics. We also investigated and compared the feasibility of testing these products for efficacy in humans.
\vspace{2cm}

\textbf{\uppercase{NYMU-Taipei}}
\FloatBarrier
\begin{table}[h]
\begin{tabular}{lp{2.5cm}llll}
\textbf{Location:} & Taiwan & \multicolumn{1}{|l}{} & \textbf{Track:}   & Environment \\
\textbf{Region:}   & Asia   & \multicolumn{1}{|l}{} & \textbf{Section:} &  \\
\textbf{Year:}     & 2015   & \multicolumn{1}{|l}{} & \textbf{Awards:}  &
\end{tabular}
\end{table}
\FloatBarrier
\noindent	\textbf{Fight the Blight} \vspace{.2cm}\\
Phytophthora infestans is the causal agent of late blight disease of several members from the Solanaceae family. Potato, the third most important food crop in the world and a source of major agricultural income in many countries, easily falls victim to P. infestans. Yet most existing approaches are ineffective and have certain drawbacks. This year, the NYMU-Taipei iGEM team creates a systematic disease control method that can prevent, detect, and cure potato late blight. Inspired by competitive inhibition in pharmacology, we designed a ligand with higher affinity to block the entrance of P. infestans effector proteins. To detect infection in the plant, we devised a soil-based microbial fuel cell (SMFC) detecting salicylic acid emission and producing oscillating current. We also characterized a new defensin to inhibit nutrient absorption and further growth of the oomycete. Our goal is to provide an easily-practiced standard procedure for anyone involved in the production line.
\vspace{2cm} $ $
\pagebreak

\textbf{\uppercase{Tec-Chihuahua}}
\FloatBarrier
\begin{table}[h]
\begin{tabular}{lp{2.5cm}llll}
\textbf{Location:} & Mexico & \multicolumn{1}{|l}{} & \textbf{Track:}   & Foundational Advance \\
\textbf{Region:}   & Latin America   & \multicolumn{1}{|l}{} & \textbf{Section:} &  \\
\textbf{Year:}     & 2015   & \multicolumn{1}{|l}{} & \textbf{Awards:}  &
\end{tabular}
\end{table}
\FloatBarrier
\noindent	\textbf{The Carbon Carriers: Cell Transformation and Transfection by Carbon Nanotubes} \vspace{.2cm}\\
The manipulation of genetic material is key to the development of synthetic biology. The introduction of genetic material into different cell types is indispensable for the creation of genetically modified organisms which provide different benefits to society. However some techniques used for gene delivery into the cells have low efficiencies, can be expensive, or use complex equipment and are complicated to do. That is why in recent years it has sought new strategies for effective transformation of cells at low cost. One of these strategies is the use of nanotechnology, which has the potential of crossing cell membranes and increase solubility, stability and bioavailability of biomolecules, thereby improving efficiency of release. Here, we intend to evaluate the efficiencies of gene delivery of DNA-CNTs in Escherichia coli cultures, embryos in early development of Bos taurus and calluses of Nicotiana tabacum and compare them with the traditional methods used in the laboratory.
\vspace{2cm}

\textbf{\uppercase{UNIK\_Copenhagen}}
\FloatBarrier
\begin{table}[h]
\begin{tabular}{lp{2.5cm}llll}
\textbf{Location:} & Denmark & \multicolumn{1}{|l}{} & \textbf{Track:}   & Environment \\
\textbf{Region:}   & Europe   & \multicolumn{1}{|l}{} & \textbf{Section:} &  \\
\textbf{Year:}     & 2015   & \multicolumn{1}{|l}{} & \textbf{Awards:}  &
\end{tabular}
\end{table}
\FloatBarrier
\noindent	\textbf{SpaceMoss: Using synthetic biology for space exploration} \vspace{.2cm}\\
Space Moss is working on the quest to colonize Mars by bringing together Astrophysics and Synthetic Biology.
The idea of Martian colonisation have captured our minds for generations. Creating a sustainable environment on Mars where humans could survive, however, is not a trivial problem. Synthetic biology could help provide a solution by creating genetically modified organisms capable of producing essential compounds for Mars-colonist survival. 
Our first step has been to make moss able to produce compounds essential for it to thrive on Mars.
We focus on an antifreeze protein, as it could help the moss to survive the extreme temperatures found on the surface of the planet.
Our second step is to produce compounds useful to colonists. Therefore, we have been working on getting it to produce resveratrol, as a proof-of-concept of medical applications.
\vspace{2cm} $ $
\pagebreak

\textbf{\uppercase{Valencia\_UPV}}
\FloatBarrier
\begin{table}[h]
\begin{tabular}{lp{2.5cm}llll}
\textbf{Location:} & Spain & \multicolumn{1}{|l}{} & \textbf{Track:}   & Information Processing \\
\textbf{Region:}   & Europe   & \multicolumn{1}{|l}{} & \textbf{Section:} &  \\
\textbf{Year:}     & 2015   & \multicolumn{1}{|l}{} & \textbf{Awards:}  &
\end{tabular}
\end{table}
\FloatBarrier
\noindent	\textbf{AladDNA} \vspace{.2cm}\\
Conventional production methods require huge and specialized infrastructures, making the establishment of new production facilities in remote locations complicated. What if we could just send information that could unfold on site? 
AladDNA is a new revolutionary system able to process genetic information and give a response based on the user’s needs just like a genie in a lamp! This system uses DNA to store information inside a plant seed, acting as a miniaturized and flexible biofactory capable of producing a myriad of bioproducts such as interferon alpha or anti-choleric vaccines. Equipped with a multiplexed-optogenetically controlled circuit, AladDNA can activate the production of different high-added value products upon the reception of external signals based on combinations of light stimuli.
AladDNA allows bioproduction in any condition avoiding prohibitive costs due to infrastructures. 
No matter where you are or what you need, just ask your wish! Because AladDNA has no frontiers!
\vspace{2cm}

\textbf{\uppercase{Waterloo}}
\FloatBarrier
\begin{table}[h]
\begin{tabular}{lp{2.5cm}llll}
\textbf{Location:} & Canada & \multicolumn{1}{|l}{} & \textbf{Track:}   & Foundational Advance \\
\textbf{Region:}   & North America   & \multicolumn{1}{|l}{} & \textbf{Section:} &  \\
\textbf{Year:}     & 2015   & \multicolumn{1}{|l}{} & \textbf{Awards:}  &
\end{tabular}
\end{table}
\FloatBarrier
\noindent	\textbf{CRISPieR: re-engineering CRISPR-Cas9 with functional applications in eukaryotic systems} \vspace{.2cm}\\
CRISPR-Cas9 is an exciting tool for synthetic biologists because it can target and edit genomes with unprecedented specificity. Our team is attempting to re-engineer CRISPR to make it more flexible and easier to use. We’re making it easy to test different sgRNA designs: restriction sites added to the sgRNA backbone allow 20 nucleotide target sequences to be swapped without excessive cloning. Additionally, we’re applying recent research on viable mutations within Cas9’s PAM-interacting domain to design (d)Cas9 variants that bind to novel PAM sites, moving towards the goal of a suite of variants that can bind any desired sequence.
We believe our re-engineered CRISPR-Cas9 will give biologists increased ability to optimize targeting in many applications. The application we chose to explore is a proof-of-concept antiviral system defending the model plant Arabidopsis against Cauliflower Mosaic Virus, which would benefit from testing a large number of possible sgRNAs in the viral genome.
\vspace{2cm} $ $
\pagebreak



\textbf{\uppercase{GDSYZX-United}}
\FloatBarrier
\begin{table}[h]
\begin{tabular}{lp{2.5cm}llll}
\textbf{Location:} & China & \multicolumn{1}{|l}{} & \textbf{Track:}   & High School \\
\textbf{Region:}   & Asia   & \multicolumn{1}{|l}{} & \textbf{Section:} &  \\
\textbf{Year:}     & 2016   & \multicolumn{1}{|l}{} & \textbf{Awards:}  &
\end{tabular}
\end{table}
\FloatBarrier
\noindent	\textbf{Super-HHL1: protecting plants from photodamage} \vspace{.2cm}\\
In tropical and subtropical areas, under high-irradiance conditions, plants must efficiently protect photosystem II (PSII) from damage. In this project, we built several genetic circles with the chloroplast protein HYPERSENSITIVE TO HIGH LIGHT1 (HHL1) which response to high light and functions in protecting PSII against photodamage. We further measured the expression efficiency of HHL1 promoted by various photosensitive promoters with high-light exposure, and named the most effective one as Super-HHL1. Moreover, we studied the potential application for Super-HHL1 agriculture and economic. Taken together, we suggest that Super-HHL1 might effectively help various plants repairing PSII under high light.
\vspace{2cm}

\textbf{\uppercase{Valencia\_UPV}}
\FloatBarrier
\begin{table}[h]
\begin{tabular}{lp{2.5cm}llll}
\textbf{Location:} & Spain & \multicolumn{1}{|l}{} & \textbf{Track:}   & Food & Nutrition \\
\textbf{Region:}   & Europe   & \multicolumn{1}{|l}{} & \textbf{Section:} &  \\
\textbf{Year:}     & 2016   & \multicolumn{1}{|l}{} & \textbf{Awards:}  &
\end{tabular}
\end{table}
\FloatBarrier
\noindent	\textbf{HYPE-IT: Hack Your Plants Editing with us} \vspace{.2cm}\\
Covering food necessities is mandatory, but resources are not sustainably exploited. Global strategies to increase food productivity and quality need to be concealed with a local perspective, providing breeders with the necessary technology to improve varieties. The aim of HYPE-IT is to decrease current technological barriers for breeding local crops using precision genome engineering, easing the gene editing process using SynBio-inspired simplified CRISPR/Cas9 tools. HYPE-IT brings along a software tool that associates crop traits with specific gene targets and designs optimal gRNAs for those targets. HYPE-IT also incorporates a modular gene circuit that serves as an in vivo gRNA testing system, ensuring appropriate gRNA choice even when no precise sequence information of local varieties is available. We aim to develop a split-Cas9 system based on viral vectors to efficiently deliver the editing machinery into the plant, and to create an affordable Labcase with the necessary laboratory equipment for HYPE-IT.
\vspace{2cm} $ $
\pagebreak

\textbf{\uppercase{Cardiff\_Wales}}
\FloatBarrier
\begin{table}[h]
\begin{tabular}{lp{2.5cm}llll}
\textbf{Location:} & United Kingdom & \multicolumn{1}{|l}{} & \textbf{Track:}   & Therapeutics \\
\textbf{Region:}   & Europe   & \multicolumn{1}{|l}{} & \textbf{Section:} &  \\
\textbf{Year:}     & 2017   & \multicolumn{1}{|l}{} & \textbf{Awards:}  &
\end{tabular}
\end{table}
\FloatBarrier
\noindent	\textbf{BenthBioFactory: Using plant synthetic biology to generate therapeutics for the treatment of Graves’ Disease} \vspace{.2cm}\\
Our project involves expressing the human thyroid stimulating hormone antagonist (TSHantag) protein in the tobacco Nicotiana benthamiana. Using golden gate cloning, we are generating transcriptional units for transient expression of TSHantag after agrobacterium-mediated transformation in tobacco leaves. TSHantag has been used to treat hyperthyroid disorders such as Graves’ Disease, but has not been produced in large amounts appropriate for therapeutic use. The TSHantag protein works by inhibiting autoimmune autoantibodies, which in turn decreases elevated thyroxine levels and reduces pathologic symptoms. 
In addition, we are expanding the set of tools available for regulating gene expression in plant synthetic biology. Currently, only a small number of regulatory elements have been introduced into the iGEM Phytobrick standard. Therefore, we are generating novel inducible promoter Phytobricks that are responsive to jasmonic acid, salicylic acid and damage associated molecular patterns (DAMPs). The efficacy of these regulatory elements will be measured using a luciferase reporter system. 
\vspace{2cm}

\textbf{\uppercase{Lanzhou}}
\FloatBarrier
\begin{table}[h]
\begin{tabular}{lp{2.5cm}llll}
\textbf{Location:} & China & \multicolumn{1}{|l}{} & \textbf{Track:}   & Environment \\
\textbf{Region:}   & Asia   & \multicolumn{1}{|l}{} & \textbf{Section:} &  \\
\textbf{Year:}     & 2017   & \multicolumn{1}{|l}{} & \textbf{Awards:}  &
\end{tabular}
\end{table}
\FloatBarrier
\noindent	\textbf{A novel method in controling weeds and pests by tandem RNA Interference} \vspace{.2cm}\\
Weeds and pests are most important damages to the crop yield in the world. Traditional ways like using herbicides and pesticides will easily cause resistance and pollution problems. In addressing this challenge we focus on RNA interference (RNAi), a novel molecular technology that used for gene knockdown, which has been showing great potential in agriculture field especially for insects control, Meanwhile we notice that many weeds are the hosts or intermediate hosts of pests. Based on these two observations, we are aimed at using synthetic biology to control weeds and pests at the same time. We selected model organism Arabidopsis as basic plant verification system which has clear genetic background and field pests Aphidoidea are chosen to be discussed as well in our experiment.
\vspace{2cm} $ $
\pagebreak

\textbf{\uppercase{Missouri\_Rolla}}
\FloatBarrier
\begin{table}[h]
\begin{tabular}{lp{2.5cm}llll}
\textbf{Location:} & United States & \multicolumn{1}{|l}{} & \textbf{Track:}   & Environment \\
\textbf{Region:}   & North America   & \multicolumn{1}{|l}{} & \textbf{Section:} &  \\
\textbf{Year:}     & 2017   & \multicolumn{1}{|l}{} & \textbf{Awards:}  &
\end{tabular}
\end{table}
\FloatBarrier
\noindent	\textbf{Detecting groundwater and soil pollutants using plant biosensors} \vspace{.2cm}\\
Plant-based biosensors have immense benefits over analytical chemistry or potentiometric techniques because they continuously sample a large volume of the environment, provide warning to laypeople, and achieve the amazing specificity and sensitivity of biomolecules. 
We are developing two approaches to biosensing contaminants with plants. Both systems are based on important developments in biosensors, namely the creation of synthetic signal transduction systems in bacteria and plants and the redesign of natural periplasmic binding proteins for the detection of new ligands. Taken together, these advances could allow a computationally-designed periplasmic binding protein which binds a contaminant of interest extracellularly to transfer the signal through a phosphorylation cascade and produce a transcriptional response. We will create circuits to implement these synthetic signal transduction systems, attempt to computationally design periplasmic binding proteins for new ligands, and test the efficacy of our two biosensing approaches.
\vspace{2cm}


\textbf{\uppercase{SECA\_NZ}}
\FloatBarrier
\begin{table}[h]
\begin{tabular}{lp{2.5cm}llll}
\textbf{Location:} & New Zealand & \multicolumn{1}{|l}{} & \textbf{Track:}   & Food & Nutrition \\
\textbf{Region:}   & Asia   & \multicolumn{1}{|l}{} & \textbf{Section:} &  \\
\textbf{Year:}     & 2017   & \multicolumn{1}{|l}{} & \textbf{Awards:}  &
\end{tabular}
\end{table}
\FloatBarrier
\noindent	\textbf{Frozen in Thyme} \vspace{.2cm}\\
With an ever-growing world population, having sustainable and reliable crops for food production
is becoming increasingly important. However, every year millions of dollars’ worth of produce is damaged, lost, or never produced because of frosts. Frost damages new shoots and buds of crop plants through the formation of ice crystals within the tissues, which rupture the surrounding cells. As a result, new plant and fruit growth is severely inhibited. Despite promising research into frost resistance mechanisms, the majority of producers still utilise costly, and often ineffective, traditional methods of frost avoidance. Our team seeks to introduce a variety of frost resistance genes into the model organisms Arabidopsis thaliana and Escherichia coli for characterisation. This will provide insight into the varying ability of frost resistance genes to protect model organisms at sub-zero temperatures, ultimately leading to the production of frost tolerant crops.
\vspace{2cm} $ $
\pagebreak

\textbf{\uppercase{Valencia\_UPV}}
\FloatBarrier
\begin{table}[h]
\begin{tabular}{lp{2.5cm}llll}
\textbf{Location:} & Spain & \multicolumn{1}{|l}{} & \textbf{Track:}   & Food & Nutrition \\
\textbf{Region:}   & Europe   & \multicolumn{1}{|l}{} & \textbf{Section:} &  \\
\textbf{Year:}     & 2017   & \multicolumn{1}{|l}{} & \textbf{Awards:}  &
\end{tabular}
\end{table}
\FloatBarrier
\noindent	\textbf{ChatterPlant} \vspace{.2cm}\\
Urban overpopulation, climate change and natural resources decrement are threatening food security. Ensuring season-less, accessible and local food production promotes a sustainable agriculture. Valencia\_UPV provides a whole new system to control plant physiology at both genetic and environmental level.
ChatterPlant is a SynBio-based solution that works as plant-human interface allowing a bidirectional communication. First, a root-specific modular optogenetic circuit enables control on plants´ endogenous gene expression (e.g flowering). Then, a sensor circuit with color coded output provides specific information of stress conditions, accelerating corrective measures. The genetic setup is complemented with a hardware device, ChatterBox, specially designed to control plant’s growth conditions.
ChatterPlant’s possibilities can be improved gathering Plant SynBio knowledge. PlantLabCo is an open-access online platform which aims to unify Plant SynBio researchers’ work. Individual results can be published, supported by a modeling software tool integrated to ease the mathematical models’ generation of genetic circuits.
\vspace{2cm}


\textbf{\uppercase{Auckland\_MOD}}
\FloatBarrier
\begin{table}[h]
\begin{tabular}{lp{2.5cm}llll}
\textbf{Location:} & New Zealand & \multicolumn{1}{|l}{} & \textbf{Track:}   & Environment \\
\textbf{Region:}   & Asia   & \multicolumn{1}{|l}{} & \textbf{Section:} &  \\
\textbf{Year:}     & 2018   & \multicolumn{1}{|l}{} & \textbf{Awards:}  &
\end{tabular}
\end{table}
\FloatBarrier
\noindent	\textbf{Improving the Farmer, Environment and Nitrogen Use Efficiency} \vspace{.2cm}\\
Environmental pollution is a pressed global issue, even in clean, green New Zealand. Maintaining clean waterways is our responsibility as kaitiaki of the land (guardians in Te Reo Māori), but agricultural practices such as excess fertiliser application and cow effluent are flooding our New Zealand soils and waterways with urea. Taking a fluxomics approach in Arabidopsis thaliana, we are overexpressing a high-affinity urea transporter (DUR3) to upregulate the uptake of urea, and glutamine synthetase (GS1) to convert the toxic metabolite ammonia into glutamine. As a result, urea is removed more readily from the soil before it’s subject to groundwater leaching or surface run-off. We predict the increase in amino acid production will enhance plant growth. Applying our model to other plants like ryegrasses will allow farmers to grow pasture or forage crops that utilize urea on the paddock more efficiently, requiring less financial investment into urea fertilisers. 
\vspace{2cm} $ $
\pagebreak

\textbf{\uppercase{BOKU-Vienna}}
\FloatBarrier
\begin{table}[h]
\begin{tabular}{lp{2.5cm}llll}
\textbf{Location:} & Austria & \multicolumn{1}{|l}{} & \textbf{Track:}   & Information Processing \\
\textbf{Region:}   & Europe   & \multicolumn{1}{|l}{} & \textbf{Section:} &  \\
\textbf{Year:}     & 2018   & \multicolumn{1}{|l}{} & \textbf{Awards:}  &
\end{tabular}
\end{table}
\FloatBarrier
\noindent	\textbf{ROBOCROP –Turning Genes ON and OFF, in Yeast and Arabidopsis through a dCas9 Toggle Switch} \vspace{.2cm}\\
Our goal, communication with eukaryotes, is achieved through the heart piece of our model, the dCas9 Toggle Switch.
This will allow switching between two stable states of gene expression. It consists of 2 gene classes which we simply call the ON and OFF genes.
One gene in each class, which is considered the primary gene, codes for a gRNA which represses the antagonistic set of genes by binding to dCas9 and further blocking transcription though CRISPR Interference.
The switch can be activated either by signal molecules binding to a receptor or directly by liposome bound gRNA that is taken up by the cell.
As a proof of concept, the ON gene contains a GFP coding sequence as a reporter gene.
Our design is very universal and has many possible applications in the lab and in agriculture, such as controlling flowering time of plants to protect them from late frost.
\vspace{2cm}

\textbf{\uppercase{Cardiff\_Wales}}
\FloatBarrier
\begin{table}[h]
\begin{tabular}{lp{2.5cm}llll}
\textbf{Location:} & United Kingdom & \multicolumn{1}{|l}{} & \textbf{Track:}   & Environment \\
\textbf{Region:}   & Europe   & \multicolumn{1}{|l}{} & \textbf{Section:} &  \\
\textbf{Year:}     & 2018   & \multicolumn{1}{|l}{} & \textbf{Awards:}  &
\end{tabular}
\end{table}
\FloatBarrier
\noindent	\textbf{RNAphid - an effective RNAi pesticide against Myzus persciae, expressed in Nicotiana benthamiana} \vspace{.2cm}\\
Aphids are crop pests globally. They feed on a massive diversity of crops and can cause tremendous economic loss for farmers by reducing crop yields and grain sizes. They damage crops directly by feeding on plant vasculature, draining essential compounds, or indirectly, as hosts of a variety of plant viruses. Current agricultural practice is to use chemical pesticides, which are unfavourable due to off-target effects, harmfulness to humans, and developing resistance of aphids. Consequently, our team has attempted to produce an effective RNAi pesticide against Myzus persicae, the most economically detrimental aphid pest worldwide. In the vasculature of Nicotiana benthamiana, we express siRNAs that affect aphid bacteriocytes, cells that enable the survival of their essential symbiont, Buchnera aphidicola. We target genes BCR3 and SP3 to do this. Finally, we expand the limited PhytoBrick registry, with several plant promoters and reporter genes.
\vspace{2cm} $ $
\pagebreak

\textbf{\uppercase{HebrewU}}
\FloatBarrier
\begin{table}[h]
\begin{tabular}{lp{2.5cm}llll}
\textbf{Location:} & Israel & \multicolumn{1}{|l}{} & \textbf{Track:}   & Environment \\
\textbf{Region:}   & Asia   & \multicolumn{1}{|l}{} & \textbf{Section:} &  \\
\textbf{Year:}     & 2018   & \multicolumn{1}{|l}{} & \textbf{Awards:}  &
\end{tabular}
\end{table}
\FloatBarrier
\noindent	\textbf{The Catalysis of Dioxin Degradation } \vspace{.2cm}\\
Dioxins, a family of chemical compounds, pose a serious threat to humans, animals, and the environment. Classified as persistent environmental pollutants, these compounds move up the food chain via bioaccumulation; consequently, they are found in very harmful concentrations by the time the reach humans. Our team has set out to engineer a metabolic pathway for the complete degradation of dioxins, and detoxification of chlorinated compounds. The pathway would involve the uptake of these pollutants and their subsequent breakdown into molecules that would enter organisms’ native metabolism. We are testing the pathway in S. cerevisiae, and have prepared expression vectors and means to engineer a multitude of plants. By deploying such pathways directly into endemic plants, our solution can be tailored to specific regions. Furthermore, because we can efficiently control plant reproduction, we can responsibly implement synthetic biology to solve this issue in a non-invasive and ecological manner. 
\vspace{2cm}


\textbf{\uppercase{HSHL}}
\FloatBarrier
\begin{table}[h]
\begin{tabular}{lp{2.5cm}llll}
\textbf{Location:} & Germany & \multicolumn{1}{|l}{} & \textbf{Track:}   & Food & Nutrition \\
\textbf{Region:}   & Europe   & \multicolumn{1}{|l}{} & \textbf{Section:} &  \\
\textbf{Year:}     & 2018   & \multicolumn{1}{|l}{} & \textbf{Awards:}  &
\end{tabular}
\end{table}
\FloatBarrier
\noindent	\textbf{Enabeling Tobacco plants to hyperaccumulate heavy metals } \vspace{.2cm}\\
Our challenge is to solve the problem of heavy metal polluted soil, especially in areas of high industrial use, such as mining. 
We enable a tabacco plant to hyperaccumulate cadmium and lead by transfering genes of arabidopsis halleri and adding other special abilities that support accumulation of heavy metals.
\vspace{2cm} $ $
\pagebreak

\textbf{\uppercase{Missouri\_Rolla}}
\FloatBarrier
\begin{table}[h]
\begin{tabular}{lp{2.5cm}llll}
\textbf{Location:} & United States & \multicolumn{1}{|l}{} & \textbf{Track:}   & Environment \\
\textbf{Region:}   & North America   & \multicolumn{1}{|l}{} & \textbf{Section:} &  \\
\textbf{Year:}     & 2018   & \multicolumn{1}{|l}{} & \textbf{Awards:}  &
\end{tabular}
\end{table}
\FloatBarrier
\noindent	\textbf{BTree} \vspace{.2cm}\\
Since the year 2002, North American ash trees have been infected with and killed by an invasive beetle species known as Emerald Ash Borers (EAB). Current methods for prevention and treatment of EAB’s are too expensive and time consuming for large scale eradication. Our proposed long term solution is to develop Ash trees that are genetically resistant to EAB’s. From a known Bacillus thuringiensis Cry8Da protein, we hope to induce mutations in the protein’s receptor binding regions to create a Bt toxin specific for EAB’s. After screening modified proteins, we will utilize leaf-specific expression of the Cry Toxin in Arabidopsis thaliana as our model system for Ash trees. This method will target EAB’s as they feed on ash leaves during adulthood. We hope to present this system for future development as a safe and effective alternative to current treatment methods used in affected areas. 
\vspace{2cm}

\textbf{\uppercase{Navarra\_BG}}
\FloatBarrier
\begin{table}[h]
\begin{tabular}{lp{2.5cm}llll}
\textbf{Location:} & Spain & \multicolumn{1}{|l}{} & \textbf{Track:}   & High School \\
\textbf{Region:}   & Europe   & \multicolumn{1}{|l}{} & \textbf{Section:} &  \\
\textbf{Year:}     & 2018   & \multicolumn{1}{|l}{} & \textbf{Awards:}  &
\end{tabular}
\end{table}
\FloatBarrier
\noindent	\textbf{BioGalaxy: a project to produce plant biofactories for an extra-terrestrial future} \vspace{.2cm}\\
In this project we propose to develop a simple and cost-effective plant-based method for production and purification of recombinant proteins. 
The system is based on the production of plants transiently expressing a target protein (TP) fused to granule-bound starch synthase (GBSS). Tissues of GBSS:TP expressing plants will be milled in an aqueous buffer and the starch granules will be purified from plant tissue-derived impurities through a series of simple centrifugation and wash/elution steps allowing the starch granules to precipitate in a highly purified form. The GBSS:TP will be engineered to contain a unique cleavage site recognized by a specific protease, enabling the TP to be separated from the GBSS into the aqueous buffer, while the GBSS remains embedded the starch granule. Once treated with the protease, the starch granules will be removed by centrifugation while the highly purified cleaved TP can be further purified using conventional downstream processing.
\vspace{2cm} $ $
\pagebreak

\textbf{\uppercase{NCHU\_Taichung}}
\FloatBarrier
\begin{table}[h]
\begin{tabular}{lp{2.5cm}llll}
\textbf{Location:} & Taiwan & \multicolumn{1}{|l}{} & \textbf{Track:}   & New Application \\
\textbf{Region:}   & Asia   & \multicolumn{1}{|l}{} & \textbf{Section:} &  \\
\textbf{Year:}     & 2018   & \multicolumn{1}{|l}{} & \textbf{Awards:}  &
\end{tabular}
\end{table}
\FloatBarrier
\noindent	\textbf{Engineered Endophyte-Assisted Phytoremediation} \vspace{.2cm}\\
Endophyte can live inside the plants and work together with them without causing harm to the host plant. With the large and deep root system of plants, the endophyte can have further impact in soil. A serious case of soil contamination is dioxin pollution after the Vietnam War. Dioxin is a group of toxic compounds that accumulate in the environment and are difficult to break down naturally. Tackle with large area soil dioxin contamination is hard, since the most efficient way to clean up is burning, which is eco-unfriendly and costly. Our project combines phytoremediation and engineered endophyte to clean dioxin-contaminated soil. We engineered an endophyte with membrane transporter, dehalogenase and laccase to intake and break down dioxin, and created biobricks compatible shuttle vector that can express in a well-researched endophyte, Burkholderia phytofirmans. This platform can potentially apply to projects that related to or benefit from plant-microbe interaction.
\vspace{2cm}


\textbf{\uppercase{SHSID\_China}}
\FloatBarrier
\begin{table}[h]
\begin{tabular}{lp{2.5cm}llll}
\textbf{Location:} & China & \multicolumn{1}{|l}{} & \textbf{Track:}   & High School \\
\textbf{Region:}   & Asia   & \multicolumn{1}{|l}{} & \textbf{Section:} &  \\
\textbf{Year:}     & 2018   & \multicolumn{1}{|l}{} & \textbf{Awards:}  &
\end{tabular}
\end{table}
\FloatBarrier
\noindent	\textbf{Everglow} \vspace{.2cm}\\
With electricity consumption increasing across the globe, the conservation of energy has become a topic of major concern. Our team has devised an innovative solution to reduce electricity usage by attempting to create genetically modified bioluminescent plants. By altering particles on the microscopic level, we hope to create plants that can glow and thus replace electricity in the future. To these ends, our team conducted experiments to transfer the lux operon, a cluster of genes (LuxCDABEG) that control bioluminescence in the bacterial species Aliivibrio fischeri, to plant species like Nicotiana tabacum. We also attempted to insert an extra copy of LuxG to enhance the effects of bioluminescence. The results are promising and point to the possibility of creating a greener alternative to current lighting. Furthermore, we will design a new plasmid that can detect potential stress factors like ethanol and report the signal with stronger bioluminescence.
\vspace{2cm} $ $
\pagebreak

\textbf{\uppercase{UGA}}
\FloatBarrier
\begin{table}[h]
\begin{tabular}{lp{2.5cm}llll}
\textbf{Location:} & United States & \multicolumn{1}{|l}{} & \textbf{Track:}   & Food & Nutrition \\
\textbf{Region:}   & North America   & \multicolumn{1}{|l}{} & \textbf{Section:} &  \\
\textbf{Year:}     & 2018   & \multicolumn{1}{|l}{} & \textbf{Awards:}  &
\end{tabular}
\end{table}
\FloatBarrier
\noindent	\textbf{Development of Gal4/UAS Reporter Systems for use in Plants} \vspace{.2cm}\\
The development of inducible expression systems in plants is imperative to the field of synthetic biology. The University of Georgia’s 2018 iGEM team is expanding the iGEM registry’s profile of plant promoters and reporters. Here we report a modified Gal4/UAS system. The Gal4/UAS system is an inducible promoter system native to yeast that utilizes the Gal4 transcription factor to activate genes downstream of a minimal promoter enhanced by an upstream activator sequence (UAS). We have created a 6X UAS repeat combined with a minimal 35S promoter to provide enhanced expression of reporter genes such as GFP, AmilC, and the apoptotic initiator from bell peppers, BS3, in the model organism, Nicotiana Benthamiana. The introduction of these expression systems to the iGEM registry will enable future iGEM teams to produce targeted expression in plants with ease using a binary vector system.
\vspace{2cm}

\textbf{\uppercase{Bonn}}
\FloatBarrier
\begin{table}[h]
\begin{tabular}{lp{2.5cm}llll}
\textbf{Location:} & Germany & \multicolumn{1}{|l}{} & \textbf{Track:}   & Energy \\
\textbf{Region:}   & Europe   & \multicolumn{1}{|l}{} & \textbf{Section:} &  \\
\textbf{Year:}     & 2019   & \multicolumn{1}{|l}{} & \textbf{Awards:}  &
\end{tabular}
\end{table}
\FloatBarrier
\noindent	\textbf{Optoplant: Lighting up your way to a better future} \vspace{.2cm}\\
Creating a plant that can glow in the dark is not a unique project; it has been tried before and not with much success, which is why we are taking a more conservative approach to this project: By testing various parts of gene constructs and bioluminescent systems we can quantify and show the best parts available for someone to make a functional glowing plant.
The parts we are testing in a bacterial chassis (E. Coli) and then in a plant chassis (Nicotiana Benthamiana) are Promoters, Mutated LuxAB Complexes, Riboswitch, and Fluorescent Reporter Genes. By using IIS Restriction we can interchange any part of a gene construct with relative ease allowing us to quickly test and compare various constructs due to the modular nature of our cloning method.
Optoplant will provide the first basis for others working with bioluminescence systems and plant engineering.
\vspace{2cm} $ $
\pagebreak




\textbf{\uppercase{GDSYZX}}
\FloatBarrier
\begin{table}[h]
\begin{tabular}{lp{2.5cm}llll}
\textbf{Location:} & China & \multicolumn{1}{|l}{} & \textbf{Track:}   & High School \\
\textbf{Region:}   & Asia   & \multicolumn{1}{|l}{} & \textbf{Section:} &  \\
\textbf{Year:}     & 2019   & \multicolumn{1}{|l}{} & \textbf{Awards:}  &
\end{tabular}
\end{table}
\FloatBarrier
\noindent	\textbf{Adorabal(Salidroside produced in  Arabidopsis thaliana)} \vspace{.2cm}\\
The rhizomes and roots of Rhodiola rosea have been used for centuries for medicinal purposes.Recent interest in the species Rhodiola rosea in the West arose from the use of the rhizome as an adaptogen for the treatment of stress, but in the last few years, chemical and pharmacological studies have confirmed other valuable medicinal properties.
Approaches on biosynthesis of salidroside in Rhodiola rosea and its key metabolic enzymes have been published, and the required precursor substance exist in Nicotiana benthamiana have been found. Arabidopsis thaliana has the potential of synthetizing salidroside which worth researching.
Hence, we were inspired to combine the key metabolic enzymes and these two plants, which are further more competent in commercial production.
Our project aims to use the techniques of synthetic biology to provide a sustainable way to obtain large quantities of salidroside in arabidopsis protoplasts.
\vspace{2cm}

\textbf{\uppercase{Nanjing}}
\FloatBarrier
\begin{table}[h]
\begin{tabular}{lp{2.5cm}llll}
\textbf{Location:} & China & \multicolumn{1}{|l}{} & \textbf{Track:}   & High School \\
\textbf{Region:}   & Asia   & \multicolumn{1}{|l}{} & \textbf{Section:} &  \\
\textbf{Year:}     & 2019   & \multicolumn{1}{|l}{} & \textbf{Awards:}  &
\end{tabular}
\end{table}
\FloatBarrier
\noindent	\textbf{Anti-Aphid Angiosperm: use cotton chitinase gene to resist pest invasion} \vspace{.2cm}\\
Our project uses agrobacteria to produce chitinase in tobacco leaves in order to help resist insect’s infection. Chitin is the composition of insects’ exoskeleton and digestion tract. Chitinase can decompose chitin and hence reduce insect infection. This method can be used as a “green pesticide” which doesn’t damage the environment. The agrobacteria can transfer part of its plasmid into plant genome, which is the T-DNA. We inserted the chitinase gene into the vector pCAMBIA 2301 and adopted a binary system. We first let the vector amplify in E.coli DH5α. Then, we extracted the plasmids and inserted them into agrobacteria tumefacien GV3101. The bacteria can be injected into tobacco leaves and contribute to a brand new ability to defend against pests.
\vspace{2cm} $ $
\pagebreak

\textbf{\uppercase{Navarra\_BG}}
\FloatBarrier
\begin{table}[h]
\begin{tabular}{lp{2.5cm}llll}
\textbf{Location:} & Spain & \multicolumn{1}{|l}{} & \textbf{Track:}   & High School \\
\textbf{Region:}   & Europe   & \multicolumn{1}{|l}{} & \textbf{Section:} &  \\
\textbf{Year:}     & 2019   & \multicolumn{1}{|l}{} & \textbf{Awards:}  &
\end{tabular}
\end{table}
\FloatBarrier
\noindent	\textbf{Biogalaxy2: a project to produce plant biofactories for an extra-terrestrial environment.} \vspace{.2cm}\\
In a previous project we developed a simple and cost-effective plant-based method for production and purification of recombinant proteins. The system was based on the production of “GBSS::TP” plants transiently expressing a target protein (TP) fused to granule-bound starch synthase (GBSS) containing a unique cleavage site recognized by a specific protease that enables the TP to be separated from the GBSS into an aqueous buffer, while the GBSS remains embedded the starch granule. The cleaved TP can be highly purified upon a single and simple centrifugation step of protease-treated
plant tissues. 
The aim of this project is to improve the technology by producing plants stably expressing GBSS::TP that are capable of growing under challenging conditions of low gravity, high irradiance, etc. occurring in extra-terrestrial environments. 
The project involves the collaboration with the European Space Agency (ESA) and the Spanish National Research Council (CSIC).
\vspace{2cm}


\textbf{\uppercase{Stony\_Brook}}
\FloatBarrier
\begin{table}[h]
\begin{tabular}{lp{2.5cm}llll}
\textbf{Location:} & United States & \multicolumn{1}{|l}{} & \textbf{Track:}   & Environment \\
\textbf{Region:}   & North America   & \multicolumn{1}{|l}{} & \textbf{Section:} &  \\
\textbf{Year:}     & 2019   & \multicolumn{1}{|l}{} & \textbf{Awards:}  &
\end{tabular}
\end{table}
\FloatBarrier
\noindent	\textbf{Potential prevention of TMV mottling and necrosis via yeast XRN1 gene expression in plants} \vspace{.2cm}\\
Despite being coined the Tobacco Mosaic Virus, TMV is known to infect over 350 different species of plants around the globe, threatening crop yields for dependant farmers.  Because the virus is spread between plants via pollinators such as bees, the preventative solution has been to use pesticides to avoid interaction between the bees and the affected crops.  To alleviate the spread of TMV while simultaneously preserving environmental safety, we looked into expressing the yeast gene, XRN1 in plants.  By producing the protein XRN1-p, yeast has a means of breaking down non-local and invasive RNA, a system that the eukaryotic N. Benthamiana does not have.  Ultimately, by expressing the yeast gene in our tobacco plants, we hope to both test whether it would successfully breakdown the viral RNA while also exploring whether yeast gene expression in plants is viable.
\vspace{2cm} $ $
\pagebreak



\textbf{\uppercase{Duesseldorf}}
\FloatBarrier
\begin{table}[h]
\begin{tabular}{lp{2.5cm}llll}
\textbf{Location:} & Germany & \multicolumn{1}{|l}{} & \textbf{Track:}   & Environment \\
\textbf{Region:}   & Europe   & \multicolumn{1}{|l}{} & \textbf{Section:} &  \\
\textbf{Year:}     & 2020   & \multicolumn{1}{|l}{} & \textbf{Awards:}  &
\end{tabular}
\end{table}
\FloatBarrier
\noindent	\textbf{Mossphate: Yesterday's wastewater can fuel tomorrow's crops} \vspace{.2cm}\\
Phosphate is an essential element that fulfills diverse cellular functions in all living organisms. It is a key limiting factor of plant growth and therefore used for the production of fertilizers. However, phosphate is a non-renewable resource and its natural reserves are dramatically decreasing, while the growing world population has 
led to a growing demand of phosphate fertilizers.

Our project is to accumulate phosphate from wastewater and reuse it for the agricultural sector using the moss Physcomitrella patens. The moss has been genetically engineered to accumulate phosphate in the form of polyphosphate granules through the introduction of polyphosphate kinases and additional phosphate transporters.

With these modifications, we hope to provide a sustainable way to filter phosphate from wastewater and grow phosphate-rich moss plants. These mosses can be directly used as fertilizer to provide crops with recycled phosphate.
\vspace{2cm}

\textbf{\uppercase{Hamburg}}
\FloatBarrier
\begin{table}[h]
\begin{tabular}{lp{2.5cm}llll}
\textbf{Location:} & Germany & \multicolumn{1}{|l}{} & \textbf{Track:}   & Food & Nutrition \\
\textbf{Region:}   & Europe   & \multicolumn{1}{|l}{} & \textbf{Section:} &  \\
\textbf{Year:}     & 2020   & \multicolumn{1}{|l}{} & \textbf{Awards:}  &
\end{tabular}
\end{table}
\FloatBarrier
\noindent	\textbf{Unicorn - an aMAIZEing concept} \vspace{.2cm}\\
Synthetic biologists imagine fancy synthetic circuits, which perform well in silico but have unforeseen effects in applications. In many cases we add synthetic control instruments (e.g. promoters) to already complicated cells, hoping everything works as planned even in the cellular context. But this additional complexity can pose a problem, due to unknown interactions with regulatory processes. Our aim is to make synthetic gene control easier. Cells possess specific responses to external stimuli, like pathogen infection. We propose a new mechanism which connects the natural cell response with the reliable expression of a target output, making synthetic gene control less complex and more replicable. After the proof of concept in E. coli, Zea mays and Arabidopsis thaliana we hope that our universally applicable method will be used to create sustainable pathogen and disease resistances in crop plants. 
\vspace{2cm} $ $
\pagebreak

\textbf{\uppercase{Stony\_Brook}}
\FloatBarrier
\begin{table}[h]
\begin{tabular}{lp{2.5cm}llll}
\textbf{Location:} & United States & \multicolumn{1}{|l}{} & \textbf{Track:}   & Environment \\
\textbf{Region:}   & North America   & \multicolumn{1}{|l}{} & \textbf{Section:} &  \\
\textbf{Year:}     & 2020   & \multicolumn{1}{|l}{} & \textbf{Awards:}  &
\end{tabular}
\end{table}
\FloatBarrier
\noindent	\textbf{Light-triggered knockdown of the WUSCHEL gene in Nicotiana benthamiana} \vspace{.2cm}\\
Genetically modified (GM) crops have seen widespread adoption in large-scale agriculture, given their potential to improve commercial farming yields and mitigate crop losses from pests and pathogens. With widespread adoption, they could also increase the incidence of gene flow—the transfer of genetic variation across populations—from transgenic to wild crops, threatening biodiversity. Hence, a solution is proposed wherein an optogenetic killswitch, introduced in Nicotiana benthamiana, preventing plant development upon exposure to UV-B light (~311 nm). Through the optogenetic pair comprised of ULTRAVIOLET RESPONSE LOCUS 8 (UVR8) with attached tetracycline repressor domain (TetR) and CONSTITUTIVE PHOTOMORPHOGENIC1 (COP1) with attached VP16 transactivation domain, the transcription of synthetic trans-acting small interfering RNAs (syn-tasiRNAs) will be controlled. These syn-tasiRNAs will disrupt the CLAVATA-WUSCHEL signaling pathway through the knockdown of the WUSCHEL (WUS) gene. Stem cells in the shoot apical meristem (SAM) will differentiate, causing stem cell depletion and prevention of further plant growth.

